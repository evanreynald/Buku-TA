\begin{center}
  \large\textbf{ABSTRAK}
\end{center}

\addcontentsline{toc}{chapter}{ABSTRAK}

\vspace{2ex}

\begingroup
% Menghilangkan padding
\setlength{\tabcolsep}{0pt}

\noindent
\begin{tabularx}{\textwidth}{l >{\centering}m{2em} X}
  Nama Mahasiswa    & : & \name{}         \\

  Judul Tugas Akhir & : & \tatitle{}      \\

  Pembimbing        & : & 1. \advisor{}   \\
                    &   & 2. \coadvisor{} \\
\end{tabularx}
\endgroup

% Ubah paragraf berikut dengan abstrak dari tugas akhir
Penelitian ini menghadirkan pendekatan baru untuk meningkatkan mobilitas bagi pasien ALS dengan mengembangkan sistem kontrol pergerakan kursi roda menggunakan CNN \emph{(Convolutional Neural Network)}, yang berfokus pada pengenalan pose mata, dan diimplementasikan pada platform Intel NUC. Studi ini didorong oleh tantangan yang dihadapi oleh pasien ALS, yang meskipun kehilangan mobilitas fisik, tetap dapat menggerakkan mata. Sistem yang diusulkan bertujuan untuk memberikan kemandirian yang lebih besar dan peningkatan kualitas hidup bagi individu tersebut. Metodologi penelitian ini terdiri dari beberapa komponen kunci: pengambilan dan pengolahan gambar mata, ekstraksi fitur yang relevan, estimasi dan klasifikasi posisi mata, serta eksekusi sistem kontrol pada \textit{Next Unit of Computing (NUC)}. Dengan memanfaatkan MediaPipe untuk pengenalan pose secara \emph{real-time} dan kemampuan komputasi Intel NUC, studi ini menghasilkan peningkatan yang signifikan dalam otonomi pengguna kursi roda dengan gangguan mobilitas.  Model klasifikasi yang digunakan menunjukkan kinerja yang sangat baik berdasarkan hasil evaluasi \emph{confusion matrix}, dengan nilai \emph{accuracy, precision, recall, dan f-1 score} sebesar 99\%. Performa model pada jarak 30 dan 50 cm, model memiliki akurasi tertinggi yaitu 100\%. Model memiliki kinerja yang cukup baik di berbagai tingkat pencahayaan, dengan akurasi tertinggi 100\% pada pencahayaan 131 Lux. Model juga menunjukkan hasil deteksi yang cukup baik pada subjek yang berbeda-beda. Sistem menunjukkan kinerja FPS yang cukup baik pada NUC. Waktu respons motor rata-rata untuk perintah "Kanan," "Kiri," dan "Mundur" di bawah 0,25 detik, sedangkan untuk perintah "Maju" dan "Stop" sekitar 0,43 detik. Motor kursi roda menunjukkan waktu output yang konsisten di setiap kelas perintah, dengan standar deviasi terendah pada perintah "Maju" (0,117), menunjukkan keandalan sistem dalam memberikan respons yang stabil dan konsisten. Penelitian ini tidak hanya berkontribusi pada bidang teknologi bantu, tetapi juga berpotensi menjadi model untuk inovasi masa depan dalam solusi mobilitas bagi individu dengan berbagai disabilitas fisik.

% Ubah kata-kata berikut dengan kata kunci dari tugas akhir
Kata Kunci: \emph{Teknologi Bantu, Pengenalan Pose, Kontrol Kursi Roda, CNN, MediaPipe}