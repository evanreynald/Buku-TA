\chapter{PENUTUP}
\label{chap:penutup}

% Ubah bagian-bagian berikut dengan isi dari penutup

\section{Kesimpulan}
\label{sec:kesimpulan}

Berdasarkan hasil pengujian yang telah dilakukan dan dianalisa pada bab sebelumnya, didapatkan beberapa kesimpulan sebagai berikut:

\begin{enumerate}[nolistsep]

  \item Penelitian ini berhasil mengembangkan sistem kontrol kursi roda berbasis pose mata menggunakan CNN. Sistem ini mampu mengenali pose mata dan menerjemahkannya menjadi perintah untuk mengontrol kursi roda. Sistem ini menawarkan solusi inovatif bagi pasien ALS yang hanya dapat menggerakkan mata.

  \item Model klasifikasi yang digunakan menunjukkan kinerja yang sangat baik berdasarkan hasil evaluasi \emph{confusion matrix} dengan nilai \emph{accuracy, precision, recall, dan f-1 score} sebesar 99\%. Model dapat mengenali kelima pose mata dengan akurat.

  \item Pada jarak 30 dan 50 cm, model memiliki akurasi tertinggi yaitu 100\%. Pengujian menunjukkan bahwa semakin dekat jarak mata dengan kamera, maka semakin tinggi akurasi yang dihasilkan.
  
  \item Model memiliki kinerja yang cukup baik di berbagai tingkat pencahayaan, dengan akurasi tertinggi 100\% pada pencahayaan 131 Lux. Hal ini menunjukkan bahwa cahaya memiliki pengaruh yang signifikan terhadap akurasi model, dengan peningkatan akurasi pada pencahayaan yang lebih tinggi.
  
  \item Hasil pengujian performa FPS menunjukkan bahwa sistem dapat mempertahankan FPS yang lebih stabil dan lebih cepat pada laptop. Hal ini disebabkan perbedaaan spesifikasi yaitu laptop menggunakan GPU, sedangkan NUC menggunakan CPU.

  \item Waktu respons rata-rata motor untuk perintah "Kanan," "Kiri," dan "Mundur" di bawah 0,25 detik, menunjukkan sistem yang responsif. Sedangkan untuk perintah "Maju" adalah 0,4337 detik dan untuk perintah "Stop" adalah 0,4318, menunjukkan waktu respons yang lebih tinggi karena perintah memerlukan waktu yang lebih lama untuk dieksekusi.
  
  \item Motor kursi roda memiliki waktu output yang konsisten di setiap kelas perintah, dengan variabilitas yang relatif sempit dengan nilai standar deviasi untuk kelas "Kanan" sebesar 0,408, kelas "Kiri" sebesar 0,371, kelas "Maju" sebesar 0,117, kelas "Mundur" sebesar 0,156, dan kelas "Stop" sebesar 0,361. Di antara semua kelas perintah, kelas maju memiliki nilai yang paling stabil.

\end{enumerate}

\section{Saran}
\label{chap:saran}

Untuk pengembangan lebih lanjut pada penelitian ini, dapat diberikan beberapa saran yang bisa dipertimbangkan sebagai berikut:

\begin{enumerate}[nolistsep]

  \item Mengoptimalkan dataset pelatihan, terutama untuk label kelas "Maju" dan "Mundur", dengan menerapkan augmentasi pose mata pada citra pelatihan yang akan digunakan.
  
  \item Variasi FPS pada Intel NUC menunjukkan adanya peluang untuk optimalisasi performa. Disarankan untuk meningkatkan efisiensi pemrosesan model dan alokasi sumber daya pada Intel NUC.

\end{enumerate}