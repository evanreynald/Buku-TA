\begin{center}
  \large\textbf{ABSTRACT}
\end{center}

\addcontentsline{toc}{chapter}{ABSTRACT}

\vspace{2ex}

\begingroup
% Menghilangkan padding
\setlength{\tabcolsep}{0pt}

\noindent
\begin{tabularx}{\textwidth}{l >{\centering}m{3em} X}
  \emph{Name}     & : & \name{}         \\

  \emph{Title}    & : & \engtatitle{}   \\

  \emph{Advisors} & : & 1. \advisor{}   \\
                  &   & 2. \coadvisor{} \\
\end{tabularx}
\endgroup

% Ubah paragraf berikut dengan abstrak dari tugas akhir dalam Bahasa Inggris
\emph{This research presents a novel approach to improving mobility for ALS patients by developing a wheelchair movement control system using CNN (Convolutional Neural Network), focusing on eye pose recognition, and implementing it on the Intel NUC platform. This study is motivated by the challenges faced by ALS patients, who, despite losing physical mobility, can still move their eyes. The proposed system aims to provide greater independence and improved quality of life for these individuals. The research methodology comprises several key components: eye image capture and processing, relevant feature extraction, eye position estimation and classification, and control system execution on the Next Unit of Computing (NUC). Utilizing MediaPipe for real-time pose recognition and the computational capabilities of Intel NUC, this study significantly enhances the autonomy of wheelchair users with mobility impairments. The classification model used demonstrates excellent performance based on the evaluation of the confusion matrix, with accuracy, precision, recall, and F-1 score values of 99\%. It achieves the highest accuracy of 100\% at distances of 30 and 50 cm. It performs well under various lighting conditions, with the highest accuracy of 100\% at 131 Lux. It also shows good detection results across different subjects. The system shows decent FPS performance on the NUC. The average motor response time for "Right," "Left," and "Backward" commands is below 0.25 seconds, while for "Forward" and "Stop" commands it is around 0.43 seconds. The wheelchair motor demonstrates consistent output time across all command classes, with the lowest standard deviation on the "Forward" command (0.117), indicating the system's reliability in providing stable and consistent responses. This research not only contributes to the field of assistive technology but also has the potential to serve as a model for future innovations in mobility solutions for individuals with various physical disabilities.}

% Ubah kata-kata berikut dengan kata kunci dari tugas akhir dalam Bahasa Inggris
\emph{Keywords}: \emph{Assistive Technology, Gesture Recognition, Wheelchair Control, CNN, MediaPipe}