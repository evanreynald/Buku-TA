\begin{center}
  \large\textbf{ABSTRACT}
\end{center}

\addcontentsline{toc}{chapter}{ABSTRACT}

\vspace{2ex}

\begingroup
% Menghilangkan padding
\setlength{\tabcolsep}{0pt}

\noindent
\begin{tabularx}{\textwidth}{l >{\centering}m{3em} X}
  \emph{Name}     & : & \name{}         \\

  \emph{Title}    & : & \engtatitle{}   \\

  \emph{Advisors} & : & 1. \advisor{}   \\
                  &   & 2. \coadvisor{} \\
\end{tabularx}
\endgroup

% Ubah paragraf berikut dengan abstrak dari tugas akhir dalam Bahasa Inggris
\emph{This thesis presents a novel approach to enhancing mobility for ALS patients by developing a wheelchair movement control system using MediaPipe, centered on eye gesture recognition, and implemented on a Jetson Nano platform. The study is motivated by the challenges faced by ALS patients who, despite losing physical mobility, retain the ability to move their eyes. The proposed system aims to provide these individuals with increased independence and improved quality of life. The methodology comprises several key components: capturing and processing eye images, extracting relevant features, estimating and classifying eye positions, and executing the control system on a Next Unit of Computing (NUC). The integration of computer vision technology with an embedded system is a cornerstone of this research, ensuring a responsive and efficient wheelchair control mechanism. By leveraging the power of MediaPipe for real-time gesture recognition and the computational capabilities of Jetson Nano, the study promises to significantly enhance the autonomy of wheelchair users with mobility impairments. This research not only contributes to the field of assistive technology but also serves as a potential model for future innovations in mobility solutions for individuals with various physical disabilities.}

% Ubah kata-kata berikut dengan kata kunci dari tugas akhir dalam Bahasa Inggris
\emph{Keywords}: \emph{Assistive Technology, Gesture Recognition, Wheelchair Control}.
