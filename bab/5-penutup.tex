\chapter{PENUTUP}
\label{chap:penutup}

% Ubah bagian-bagian berikut dengan isi dari penutup

\section{Kesimpulan}
\label{sec:kesimpulan}

Berdasarkan hasil pengujian yang telah dilakukan dan dianalisa pada bab sebelumnya, didapatkan beberapa kesimpulan sebagai berikut:

\begin{enumerate}[nolistsep]

  \item Penelitian ini berhasil mengembangkan sistem kontrol kursi roda berbasis gestur mata menggunakan MediaPipe dan Intel NUC. Sistem ini mampu mengenali gerakan mata dan menerjemahkannya menjadi perintah untuk mengontrol kursi roda. Sistem ini menawarkan solusi inovatif bagi pasien ALS yang hanya dapat menggerakkan mata.

  \item Model klasifikasi yang digunakan menunjukkan kinerja yang sangat baik berdasarkan hasil evaluasi confusion matrix. Model dapat mengenali berbagai gerakan mata dengan cepat dan konsisten, memastikan kontrol kursi roda yang andal.

  \item Model tetap memiliki performa yang stabil pada jarak 30 hingga 90 cm. Pada jarak 30 dan 50 cm, model memiliki akurasi tertinggi yaitu 100\%. Model menunjukkan bahwa semakin dekat jarak mata dengan kamera, semakin tinggi akurasi yang dihasilkan.
  
  \item Model memiliki kinerja yang cukup baik di berbagai tingkat pencahayaan, dengan akurasi tertinggi 100\% pada pencahayaan 131 Lux. Hal ini menunjukkan bahwa cahaya memiliki pengaruh yang signifikan terhadap akurasi model, dengan peningkatan akurasi pada pencahayaan yang lebih tinggi.
  
  \item Pengujian performa FPS menunjukkan bahwa sistem dapat mempertahankan FPS yang stabil pada laptop (10,415 - 13,972 FPS) dan Intel NUC (8,405 - 12,448 FPS). Variasi ini menunjukkan bahwa sistem dapat berjalan dengan baik pada kedua perangkat, meskipun dengan performa dan kestabilan yang berbeda.
  
  \item Waktu respons rata-rata motor untuk perintah "Kanan," "Kiri," dan "Mundur" di bawah 0,25 detik, menunjukkan sistem yang responsif. Sedangkan untuk perintah "Maju" adalah 0,4337 detik dan untuk perintah "Stop" adalah 0,4318, menunjukkan waktu respons yang lebih tinggi karena perintah tersebut memerlukan waktu yang lebih lama untuk dieksekusi.
  
  \item Motor kursi roda memiliki waktu output yang konsisten di setiap kelas perintah, dengan variabilitas yang relatif sempit dengan nilai standar deviasi untuk kelas "Kanan" sebesar 0,408, kelas "Kiri" sebesar 0,371, kelas "Maju" sebesar 0,117, kelas "Mundur" sebesar 0,156, dan kelas "Stop" sebesar 0,361. Di antara semua kelas perintah, kelas maju memiliki nilai yang paling stabil, menunjukkan keandalan sistem dalam memberikan respons stabil dan konsisten saat perintah maju diberikan.

\end{enumerate}

\section{Saran}
\label{chap:saran}

Untuk pengembangan lebih lanjut pada penelitian ini, dapat diberikan beberapa saran yang bisa dipertimbangkan sebagai berikut:

\begin{enumerate}[nolistsep]

  \item Mengoptimalkan dataset pelatihan, terutama untuk label kelas maju dan mundur, dengan menerapkan augmentasi gestur mata pada seluruh citra pelatihan yang akan digunakan.
  
  \item Variasi FPS pada Intel NUC menunjukkan adanya peluang untuk optimalisasi performa. Disarankan untuk meningkatkan efisiensi pemrosesan model dan alokasi sumber daya pada Intel NUC.

\end{enumerate}