\begin{center}
  \large\textbf{ABSTRACT}
\end{center}

\addcontentsline{toc}{chapter}{ABSTRACT}

\vspace{2ex}

\begingroup
% Menghilangkan padding
\setlength{\tabcolsep}{0pt}

\noindent
\begin{tabularx}{\textwidth}{l >{\centering}m{3em} X}
  \emph{Name}     & : & \name{}         \\

  \emph{Title}    & : & \engtatitle{}   \\

  \emph{Advisors} & : & 1. \advisor{}   \\
                  &   & 2. \coadvisor{} \\
\end{tabularx}
\endgroup

% Ubah paragraf berikut dengan abstrak dari tugas akhir dalam Bahasa Inggris
\emph{This research presents a new approach to enhance mobility for ALS patients by developing a wheelchair movement control system using CNN (Convolutional Neural Network), which focuses on eye movement recognition, and is implemented on the Intel NUC platform. The study is driven by the challenges faced by ALS patients, who, despite losing physical mobility, can still move their eyes. The proposed system aims to provide greater independence and improved quality of life for these individuals. The research methodology consists of several key components: eye image capture and processing, extraction of relevant features, eye position estimation and classification, and control system execution on the Next Unit of Computing (NUC). By utilizing MediaPipe for real-time pose recognition and the computational capabilities of the Intel NUC, this study promises significant improvements in the autonomy of wheelchair users with mobility impairments. The classification model shows excellent performance based on the evaluation results of the confusion matrix, with accuracy, precision, recall, and F-1 score of 100\%. The performance of the model at a distance of 30 and 50 cm, the model has the highest accuracy of 100\%. The model performed reasonably well across various lighting levels, with the highest accuracy of 100\% at 131 Lux lighting. The system maintained stable FPS on both laptops and Intel NUCs, showing better performance on laptops. The average motor response time for “Right,” “Left,” and “Back” commands was below 0.25 seconds, while for “Forward” and “Stop” commands it was around 0.43 seconds. The wheelchair motors showed consistent output times in each command class, with the lowest standard deviation in the “Forward” command (0.117), demonstrating the reliability of the system in providing stable and consistent responses. This research not only contributes to the field of assistive technology, but also has the potential to serve as a model for future innovations in mobility solutions for individuals with various physical disabilities.}

% Ubah kata-kata berikut dengan kata kunci dari tugas akhir dalam Bahasa Inggris
\emph{Keywords}: \emph{Assistive Technology, Gesture Recognition, Wheelchair Control, CNN, MediaPipe}