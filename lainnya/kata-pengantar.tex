\begin{center}
  \Large
  \textbf{KATA PENGANTAR}
\end{center}

\addcontentsline{toc}{chapter}{KATA PENGANTAR}

\vspace{2ex}

% Ubah paragraf-paragraf berikut dengan isi dari kata pengantar

Puji dan syukur kehadirat Tuhan Yang Maha Esa, atas segala rahmat dan karunia-Nya,
sehingga penulis dapat menyelesaikan penelitian ini yang berjudul
"\tatitle".

Penelitian ini disusun dalam rangka pemenuhan Tugas Akhir sebagai syarat kelulusan sehingga mendapatkan gelar Sarjana Teknik komputer ITS. Oleh karena itu, penulis mengucapkan banyak terima kasih kepada:

\begin{enumerate}[nolistsep]
  \item Tuhan Yang Maha Esa atas segala Rahmat dan Berkat-Nya sehingga penulis dapat menyelesaikan masa kuliah dan mengerjakan tugas akhir ini dengan baik dan lancar,

  \item Kedua Orang Tua, Papi dan Mami, yang selalu memberikan doa, dukungan, nasihat, kepercayaan, dan uang saku, serta adik-adik dan keluarga yang telah memberi penulis motivasi dan semangat berjuang,

  \item Bapak Dr.Supeno Mardi Susiko Nugroho, ST.,MT, selaku Kepala Departemen Teknik Komputer, Fakultas Teknologi Elektro dan Informatika Cerdas, Institut Teknologi Sepuluh Nopember,

  \item Bapak Dr. Eko Mulyanto Yuniarno, S.T., M.T. selaku Dosen Pembimbing I yang telah memberikan topik, metodologi, bimbingan, wawasan, dan kesempatan untuk mengerjakan tugas akhir ini,

  \item Bapak Dion Hayu Fandiantoro, S.T., M.Eng. selaku Dosen Pembimbing II yang telah memberikan banyak masukan dan revisi seputar pengujian dan penulisan buku tugas akhir ini,

  \item Bapak-Ibu dosen pengajar Departemen Teknik Komputer, atas ilmu dan pengajaran yang telah diberikan kepada penulis selama ini,
  
  \item Teman - teman sesama bimbingan tugas akhir, terutama kursi roda, teman-teman dari lab B300, B201, dan B401 serta teman - teman Departemen Teknik Komputer lainnya yang telah membantu penulis dalam menjalani masa kuliah.

\end{enumerate}

Akhir kata, semoga penelitian ini bermanfaat bagi banyak pihak. Penulis menyadari bahwa skripsi ini masih jauh dari sempurna karena keterbatasan ilmu. Oleh karena itu, penulis sangat mengharapkan saran dan kritik yang membangun agar hasilnya bisa lebih baik di masa mendatang.

\begin{flushright}
  \begin{tabular}[b]{c}
    \place{}, \MONTH{} \the\year{} \\
    \\
    \\
    \\
    \\
    \name{}
  \end{tabular}
\end{flushright}