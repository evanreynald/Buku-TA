\chapter{PENGUJIAN DAN ANALISIS}
\label{chap:pengujiananalisis}

% Ubah bagian-bagian berikut dengan isi dari pengujian dan analisis

Pada bab ini, dipaparkan mengenai data hasil pengujian beserta analisis dari setiap pelaksanaan yang dilakukan berdasarkan tahapan yang telah dijelaskan pada bab sebelumnya mengenai metodologi serta dipaparkan juga mengenai beberapa skenario pengujian yang telah diterapkan. Skenario pengujian yang dilakukan berguna untuk mengetahui performa dan tingkat akurasi dari model beserta sistem yang telah dibuat berdasarkan tahapan yang dijelaskan pada metodologi. Skenario pengujian yang diterapkan meliputi beberapa poin sebagai berikut:

\begin{enumerate}[topsep=8pt,itemsep=4pt,partopsep=4pt, parsep=4pt]
  \item Pengujian Performa Model dengan menggunakan \emph{Confusion Matrix}
  \item Pengujian Performa Model dengan menggunakan Variasi Jarak
  \item Pengujian Performa Model dengan menggunakan Variasi Pencahayaan
  \item Pengujian Performa \emph{Frame Per Second} (FPS) pada Sistem Kontrol Kursi Roda
  \item Pengujian Waktu dari \emph{Inference Time} pada Model ke \emph{Response Time} pada Motor Kursi Roda
  \item Pengujian Kestabilan pada Motor Kursi Roda
\end{enumerate}

Penerapan metodologi beserta skenario pengujian yang dijelaskan dan dianalisa pada bab ini diharapkan dapat memberikan pemahaman terkait hasil akhir berupa kesimpulan dan saran yang didapat sehingga penelitian ini dapat dievaluasi dan terus dikembangkan.

\section{Pengujian}
\label{sec:skenariopengujian}

Pada sub-bab ini, dipaparkan mengenai hasil dari skenario pengujian yang telah dilakukan terhadap model, motor kursi roda, beserta keseluruhan sistem kontrol kursi roda. Hasil pengujian yang didapatkan berupa data yang kemudian dapat dianalisa dan diolah sehingga diketahui performa dan tingkat akurasi yang dapat digunakan sebagai dasar untuk pengembangan penelitian lebih lanjut serta menarik kesimpulan dan saran secara keseluruhan. Pengujian yang dilakukan dijelaskan sebagai berikut:

\section{Pengujian Performa Model dengan menggunakan \emph{Confusion Matrix}}

Pada penelitian ini digunakan metode \emph{Convolutional Neural Network} (CNN) untuk membuat model. Selama pembuatan model CNN, dilakukan pengambilan dan pengelompokan data citra yang kemudian digunakan sebagai \emph{dataset} untuk pelatihan dan validasi model CNN. Data yang dilatih berupa citra \emph{landmark} mata yang telah dikelompokkan menjadi lima kelas seperti yang telah dipaparkan pada metodologi. Lima kelas yang mengelompokkan data citra \emph{landmark} mata adalah 'Kanan', 'Kiri', 'Maju', 'Mundur', dan 'Stop'.

\emph{Dataset} yang digunakan dalam penelitian ini terdiri dari data citra yang berjumlah 2000 data, dimana kemudian data citra tersebut dikelompokkan berdasarkan lima kelas sehingga jumlahnya masing-masing 400 data per kelas. Data-data citra tersebut dimasukkan ke dalam lima folder dengan label sesuai dengan nama kelas yang dipaparkan sebelumnya. Selanjutnya dalam masing-masing folder kelas, 400 data citra tersebut dibagi (\emph{split}) dengan bobot 70\% untuk data \emph{training} (pelatihan) sehingga berjumlah 280 data citra yang dimasukkan pada folder \emph{training }dan bobot 30\% untuk data \emph{validation} (validasi) sehingga berjumlah 120 data citra yang dimasukkan dalam folder \emph{validation}. Kelima folder kelas tersebut akan digunakan sebagai \emph{input} data \emph{training} dan data \emph{validation} dalam pembuatan model.

Secara keseluruhan, \emph{dataset} citra \emph{landmark} mata yang digunakan dalam penelitian ini, berjumlah 2000 data citra yang dibagi ke dalam 5 kelas, dengan pembagian data pelatihan sebanyak 280 data citra dan data validasi sebanyak 120 data citra. Visualisasi pembagian \emph{dataset} yang digunakan pada penelitian ini dapat dilihat pada Gambar \ref{fig:dataset} beserta rinciannya pada Tabel \ref{tb:datadiagram}.

%Gambar 4.1
\begin{figure} [H] \centering
  % Nama dari file gambar yang diinputkan
  \includegraphics[width=1\textwidth]{gambar/bab4/visualdataset.png}
  % Keterangan gambar yang diinputkan
  \caption{Diagram Pembagian \emph{Dataset}}
  % Label referensi dari gambar yang diinputkan
  \label{fig:dataset}
\end{figure}

%Tabel 4.1
\begin{longtable}{|c|c|c|}
  \caption{\emph{Dataset} Kontrol {Kursi Roda}}
  \label{tb:datadiagram} \\
  \hline
  \rowcolor[HTML]{C0C0C0}
  \textbf{Kelas Kontrol} & \textbf{Data Pelatihan} & \textbf{Data Validasi} \\ \hline
  Kanan            & 280 Citra               & 120 Citra               \\ \hline
  Kiri              & 280 Citra               & 120 Citra               \\ \hline
  Maju             & 280 Citra               & 120 Citra               \\ \hline
  Mundur            & 280 Citra               & 120 Citra               \\ \hline
  Stop             & 280 Citra               & 120 Citra               \\ \hline
\end{longtable}

Selanjutnya, setelah proses akuisisi data telah dilakukan, model CNN dibangun dengan menggunakan \emph{dataset} pelatihan dan \emph{dataset} validasi yang sebelumnya telah diambil sebagai data pelatihan dan data pengujian. Dalam pembuatan model CNN, proses pelatihan dilakukan dengan menggunakan model CNN yang terdiri dari sebelas lapisan seperti struktur yang telah dipaparkan pada metodologi penelitian. Pelatihan pada model CNN dilakukan dengan konfigurasi 40 tahapan \emph{epoch} dengan 10 langkah pelatihan per \emph{epoch}. Ditambahkan juga modul \emph{early stopping} selama pelatihan model agar didapatkan model terbaik, yaitu pada \emph{epoch} ke-30. Berdasarkan hasil pelatihan tersebut, dihasilkan nilai akurasi sebesar 100\% dengan akurasi pengujian sebesar 99.83\%. Selain itu, hasil pelatihan juga menunjukkan nilai \emph{loss} sebesar 0.22\% dengan \emph{loss} pengujian sebesar 1.14\%. Variabel tersebut dapat dilihat pada Gambar \ref{fig:acc_loss} yang telah memvisualisasikan hasil akurasi dan akurasi pengujian serta hasil \emph{loss} dan \emph{loss} pengujian dalam bentuk grafik dengan nilai akurasi, akurasi pengujian, \emph{loss}, dan \emph{loss} pengujian terhadap tiap tahapan \emph{epoch}.

%Gambar 4.2
\begin{figure}[H]
  \centering
  \begin{subfigure}[b]{0.5\textwidth}
      \includegraphics[width=\textwidth]{gambar/bab4/model5 (30cm)/train.png}
      \caption{Akurasi}
  \end{subfigure}

  \begin{subfigure}[b]{0.5\textwidth}
      \includegraphics[width=\textwidth]{gambar/bab4/model5 (30cm)/loss.png}
      \caption{\emph{Loss}}
  \end{subfigure}
  \caption{Grafik Akurasi dan \emph{Loss} pada Hasil Pelatihan Model}
  \label{fig:acc_loss}
\end{figure}

Setelah didapatkan grafik akurasi dan \emph{loss} untuk hasil pelatihan model, selanjutnya dilakukan perhitungan untuk \emph{confusion matrix} yang dibuat berdasarkan \emph{dataset} validasi. \emph{Confusion matrix} dihitung dengan membandingkan label yang sebenarnya (\emph{true label}) dengan label yang diprediksi (\emph{predicted label}) yang kemudian kedua label tersebut akan divisualisasikan menjadi \emph{matrix} yang dibagi menjadi tiap kelas kontrol, dimana \emph{predicted label} menjadi sumbu X dan \emph{true label} menjadi sumbu Y. Dari visualisasi \emph{confusion matrix}, dapat diamati bahwa keseluruhan 600 \emph{dataset} validasi yang telah diuji berdasarkan label yang sebenarnya sudah terprediksi dengan benar.

%Gambar 4.3
\begin{figure} [H] \centering
  % Nama dari file gambar yang diinputkan
  \includegraphics[width=0.2965\textwidth]{gambar/bab4/model5 (30cm)/matrix.png}
  % Keterangan gambar yang diinputkan
  \caption{\emph{Confusion Matrix} Model Hasil Pelatihan}
  % Label referensi dari gambar yang diinputkan
  \label{fig:matrix1}
\end{figure}

Kemudian dari \emph{confusion matrix}, didapatkan hasil klasifikasi yang dapat dikelompokkan menjadi beberapa parameter yaitu \emph{true positive}, \emph{true negative}, \emph{false positive}, dan \emph{false negative}. \emph{True positive} adalah hasil dimana model memprediksi kelas positif dengan benar. Demikian pula, \emph{true negative} adalah hasil dimana model memprediksi kelas negatif dengan benar. Kelas positif adalah kelas itu sendiri, sedangkan kelas negatif adalah kelas lain selain kelas itu sendiri. \emph{False positive} adalah hasil di mana model salah memprediksi kelas positif. Dan \emph{false negative} adalah hasil di mana model salah memprediksi kelas negatif. Hasil klasifikasi dapat dilihat pada Tabel \ref{tb:cm_model} di bawah.

%Tabel 4.2
\begin{longtable}{|l|c|c|c|c|}
  \caption{Hasil klasifikasi model dengan \emph{confusion matrix}}
  \label{tb:cm_model} \\
  \hline
  \rowcolor[HTML]{C0C0C0} 
  \textbf{Kelas} & \textbf{TP} & \textbf{TN} & \textbf{FP} & \textbf{FN} \\ \hline
  Kanan    & 120          & 480         & 0           & 0           \\ \hline
  Kiri      & 120          & 480         & 0           & 0           \\ \hline
  Maju      & 120          & 480         & 0           & 0           \\ \hline
  Mundur     & 120          & 480         & 0           & 0           \\ \hline
  Stop  & 120          & 480         & 0           & 0           \\ \hline
\end{longtable}

Berdasarkan Tabel \ref{tb:cm_model} diatas, dapat dilihat pada semua kelas memiliki 120 data citra yang termasuk dalam \emph{true positive}, 480 data citra yang termasuk dalam \emph{true negative}, 0 data citra yang termasuk dalam \emph{false positive}, dan 0 data citra yang termasuk dalam \emph{false negative}. 

Dari hasil klasifikasi tersebut, selanjutnya dapat dihitung nilai validasi model. Nilai validasi yang dihitung yaitu \emph{accuracy}, \emph{precision}, \emph{recall}, dan \emph{f-1 score}. Penggambaran lebih detail dapat dilihat pada Tabel \ref{tb:vs_model}

%Tabel 4.3
\begin{longtable}{|l|c|c|c|c|}
  \caption{Hasil validasi nilai model}
  \label{tb:vs_model} \\
  \hline
  \rowcolor[HTML]{C0C0C0} 
  \textbf{Kelas} & \textbf{Accuracy} & \textbf{Precision} & \textbf{Recall} & \textbf{F1-Score} \\ \hline
  Kanan    & 100\%            & 100\%             & 100\%           & 100\%            \\ \hline
  Kiri     & 100\%          & 100\%           & 100\%           & 100\%           \\ \hline
  Maju      & 100\%          & 100\%           & 100\%          & 100\%          \\ \hline
  Mundur     & 100\%            & 100\%             & 100\%           & 100\%            \\ \hline
  Stop  & 100\%            & 100\%             & 100\%           & 100\%            \\ \hline
\end{longtable}

Dilihat dari Tabel \ref{tb:vs_model}, nilai \emph{accuracy}, \emph{precision}, \emph{recall}, dan \emph{f1-score} pada semua kelas bernilai sama. Untuk kelas 'Kanan' , didapatkan nilai \emph{accuracy} sebesar 100\%, nilai \emph{precision} sebesar 100\%, nilai \emph{recall} sebesar 100\%, dan nilai \emph{f1-score} sebesar 100\%. Kemudian untuk kelas 'Kiri' , didapatkan nilai \emph{accuracy} sebesar 100\%, nilai \emph{precision} sebesar 100\%, nilai \emph{recall} sebesar 100\%, dan nilai \emph{f1-score} sebesar 100\%. Selanjutnya pada kelas 'Maju' , didapatkan nilai \emph{accuracy} sebesar 100\%, nilai \emph{precision} sebesar 100\%, nilai \emph{recall} sebesar 100\%, dan nilai \emph{f1-score} sebesar 100\%. Untuk kelas 'Mundur' , didapatkan nilai \emph{accuracy} sebesar 100\%, nilai \emph{precision} sebesar 100\%, nilai \emph{recall} sebesar 100\%, dan nilai \emph{f1-score} sebesar 100\%. Terakhir pada kelas 'Stop' , didapatkan nilai \emph{accuracy} sebesar 100\%, nilai \emph{precision} sebesar 100\%, nilai \emph{recall} sebesar 100\%, dan nilai \emph{f1-score} sebesar 100\%. Nilai validasi tersebut bisa didapatkan karena nilai \emph{true positive}, \emph{true negative}, \emph{false positive}, dan \emph{false negative} dari \emph{confusion matrix} yang dipaparkan pada Tabel \ref{tb:cm_model} memiliki nilai prediksi yang baik untuk kelima kelas.

\section{Pengujian Performa Model dengan menggunakan Variasi Jarak}

Pada skenario pengujian selanjutnya, model diuji berdasarkan beberapa variasi jarak. Pengujian jarak berguna untuk mengetahui performa dan tingkat akurasi model jika input citra diambil dari jarak yang berbeda-beda. Hal ini perlu diuji karena semakin jauh jarak pendeteksi citra, maka semakin kecil data visual citra yang diperoleh. Variasi jarak yang digunakan adalah 30 sentimeter, 50 sentimeter, 70 sentimeter, 90 sentimeter, dan 110 sentimeter. 

Model diuji menggunakan \emph{dataset} validasi yang terdiri dari 600 data citra yang dibagi rata sebanyak 120 data citra yang dikelompokkan ke dalam lima kelas. Pengujian model dihitung dengan menggunakan \emph{confusion matrix}. Yang kemudian, dari \emph{confusion matrix} didapatkan nilai \emph{true positive} (TP), \emph{true negative} (TN), \emph{false positive} (FP), dan \emph{false negative} (FN) dari masing-masing kelas yang diuji. Selanjutnya dari keempat variabel tersebut, dapat dihitung \emph{accuracy}, \emph{precision}, \emph{recall}, dan \emph{f-1 score}. 

\subsection{Jarak 30 sentimeter}

Skenario pengujian jarak yang pertama dilakukan dengan jarak 30 sentimeter dari kamera ke objek deteksi. Berdasarkan \emph{confusion matrix} pada Gambar \ref{fig:matrix2} dapat diamati bahwa hasil pengujian dengan jarak 30 sentimeter memiliki hasil prediksi yang akurat pada semua kelas dimana semua label yang diprediksi sesuai dengan semua label yang sebenarnya.

%Gambar 4.4
\begin{figure} [H] \centering
  % Nama dari file gambar yang diinputkan
  \includegraphics[width=0.3\textwidth]{gambar/bab4/model5 (30cm)/matrix.png}
  % Keterangan gambar yang diinputkan
  \caption{\emph{Confusion Matrix} Model Pengujian dengan Jarak 30 cm}
  % Label referensi dari gambar yang diinputkan
  \label{fig:matrix2}
\end{figure}

Kemudian dari \emph{confusion matrix} yang didapat, diperoleh nilai TP, TN, FP, dan FN seperti yang tertera pada Tabel \ref{tb:cm_model2}. Dapat dilihat bahwa pada semua kelas, nilai TP yang diperoleh sebanyak 120 data citra, nilai TN sebanyak 480 data citra, nilai FP sebanyak 0 data citra, dan nilai FN sebanyak 0 data citra.

%Tabel 4.4
\begin{longtable}{|l|c|c|c|c|}
  \caption{Tabel Hasil Klasifikasi dari Jarak 30 cm}
  \label{tb:cm_model2} \\
  \hline
  \rowcolor[HTML]{C0C0C0} 
  \textbf{Kelas} & \textbf{TP} & \textbf{TN} & \textbf{FP} & \textbf{FN} \\ \hline
  Kanan    & 120          & 480         & 0           & 0           \\ \hline
  Kiri      & 120          & 480         & 0           & 0           \\ \hline
  Maju      & 120          & 480         & 0           & 0           \\ \hline
  Mundur     & 120          & 480         & 0           & 0           \\ \hline
  Stop  & 120          & 480         & 0           & 0           \\ \hline
\end{longtable}

Dari nilai TP, TN , FP, dan FN diatas, dapat dihitung nilai \emph{accuracy}, \emph{precision}, \emph{recall}, dan \emph{f-1 score} seperti yang tertera pada Tabel \ref{tb:vs_model2}. Pada semua kelas, didapat nilai \emph{accuracy} sebesar 100\%, nilai \emph{precision} sebesar 100\%, nilai \emph{recall} sebesar 100\%, dan nilai \emph{f-1 score} sebesar 100\%. Keempat variabel tersebut dapat bernilai bagus karena hasil perhitungannya berbanding lurus dengan perhitungan \emph{confusion matrix}. 

%Tabel 4.5
\begin{longtable}{|l|c|c|c|c|}
  \caption{Hasil Validasi Nilai Model dengan Jarak 30 cm}
  \label{tb:vs_model2} \\
  \hline
  \rowcolor[HTML]{C0C0C0} 
  \textbf{Kelas} & \textbf{Accuracy} & \textbf{Precision} & \textbf{Recall} & \textbf{F1-Score} \\ \hline
  Kanan    & 100\%            & 100\%             & 100\%           & 100\%            \\ \hline
  Kiri     & 100\%          & 100\%           & 100\%           & 100\%           \\ \hline
  Maju      & 100\%          & 100\%           & 100\%          & 100\%          \\ \hline
  Mundur     & 100\%            & 100\%             & 100\%           & 100\%            \\ \hline
  Stop  & 100\%            & 100\%             & 100\%           & 100\%            \\ \hline
\end{longtable}

\subsection{Jarak 50 sentimeter}

Selanjutnya, pengujian dilakukan dengan variasi jarak 50 sentimeter. Dari data pada \emph{confusion matrix}, dapat dilihat bahwa setiap kelas memiliki 120 prediksi yang benar dan tidak terdapat kesalahan klasifikasi antarkelas mana pun. Setiap nilai diagonal yang mencapai 120 mengindikasikan bahwa semua prediksi untuk kelas tersebut sesuai dengan label sebenarnya, tanpa ada satu pun yang salah. Ini menandakan bahwa model mampu dengan sempurna menginterpretasikan sinyal input untuk menghasilkan respons yang tepat sesuai dengan perintah yang diharapkan.

%Gambar 4.5
\begin{figure} [H] \centering
  % Nama dari file gambar yang diinputkan
  \includegraphics[width=0.25\textwidth]{gambar/bab4/model6 (50cm)/matrix.png}
  % Keterangan gambar yang diinputkan
  \caption{\emph{Confusion Matrix} Model Pengujian dengan Jarak 30 cm}
  % Label referensi dari gambar yang diinputkan
  \label{fig:matrix3}
\end{figure}

Pada Tabel \ref{tb:cm_model3} yang disajikan menampilkan hasil pengujian model klasifikasi yang digunakan untuk mengendalikan kursi roda melalui gestur mata dari jarak 50 sentimeter. Untuk semua kelas, angka True Positive mencatat 120, menunjukkan bahwa model secara konsisten berhasil mengidentifikasi setiap gestur dengan benar sebanyak 120 kali. True Negative, yang juga mencapai 480 untuk setiap kelas, mengindikasikan bahwa model sangat efisien dalam tidak salah mengklasifikasikan gestur lain sebagai gestur yang sedang diuji. Sementara itu, nilai untuk False Positive dan False Negative adalah nol, menunjukkan bahwa tidak ada kesalahan dalam pengklasifikasian gestur. Secara keseluruhan, tabel ini menggambarkan performa yang sangat baik dari model dalam mengklasifikasikan gestur mata dari jarak 50 sentimeter untuk pengendalian kursi roda, dengan tingkat akurasi dan keandalan yang tinggi.

%Tabel 4.6
\begin{longtable}{|l|c|c|c|c|}
  \caption{Hasil klasifikasi model dengan \emph{confusion matrix}}
  \label{tb:cm_model3} \\
  \hline
  \rowcolor[HTML]{C0C0C0} 
  \textbf{Kelas} & \textbf{TP} & \textbf{TN} & \textbf{FP} & \textbf{FN} \\ \hline
  Kanan    & 120          & 480         & 0           & 0           \\ \hline
  Kiri      & 120          & 480         & 0           & 0           \\ \hline
  Maju      & 120          & 480         & 0           & 0           \\ \hline
  Mundur     & 120          & 480         & 0           & 0           \\ \hline
  Stop  & 120          & 480         & 0           & 0           \\ \hline
\end{longtable}

Berdasarkan nilai TP, TN, FP, dan FN yang didapatkan, dapat didapatkan nilai 100\% di semua metrik evaluasi—\emph{Accuracy, Precision, Recall}, dan \emph{F1-Score}—untuk kelima kelas: "Kanan", "Kiri", "Maju", "Mundur", dan "Stop" pada jarak 50 sentimeter. Akurasi sempurna menunjukkan tidak adanya kesalahan dalam klasifikasi, sementara presisi 100\% mengindikasikan tidak ada FP, dan \emph{recall} 100\% menandakan model berhasil menangkap semua kasus positif sebenarnya tanpa kehilangan satu pun. Skor F1 yang juga sempurna menunjukkan keseimbangan optimal antara presisi dan \emph{recall}.

%Tabel 4.7
\begin{longtable}{|l|c|c|c|c|}
  \caption{Hasil validasi nilai model}
  \label{tb:vs_model3} \\
  \hline
  \rowcolor[HTML]{C0C0C0} 
  \textbf{Kelas} & \textbf{Accuracy} & \textbf{Precision} & \textbf{Recall} & \textbf{F1-Score} \\ \hline
  Kanan    & 100\%            & 100\%             & 100\%           & 100\%            \\ \hline
  Kiri     & 100\%          & 100\%           & 100\%           & 100\%           \\ \hline
  Maju      & 100\%          & 100\%           & 100\%          & 100\%          \\ \hline
  Mundur     & 100\%            & 100\%             & 100\%           & 100\%            \\ \hline
  Stop  & 100\%            & 100\%             & 100\%           & 100\%            \\ \hline
\end{longtable}

\subsection{Jarak 70 sentimeter}

Pada tahap pengujian berikutnya, dilakukan pengujian model dengan jarak 70 sentimeter. Dari \emph{confusion matrix} pada Gambar \ref{fig:matrix5}, dapat diamati performa model klasifikasi gestur dalam mengontrol pergerakan kursi roda, yang mencakup lima kelas: Kanan, Kiri, Maju, Mundur, dan Stop. Kinerja model secara keseluruhan sangat baik dengan setiap kelas kecuali 'Maju'. Untuk kelas 'Maju', model berhasil mengidentifikasi 117 kasus dengan benar namun, ada 3 kasus yang keliru diidentifikasi sebagai 'Mundur', menunjukkan sedikit kelemahan dalam membedakan antara dua gestur ini.

%Gambar 4.7
\begin{figure} [H] \centering
  % Nama dari file gambar yang diinputkan
  \includegraphics[width=0.25\textwidth]{gambar/bab4/model8 (90cm)/matrix.png}
  % Keterangan gambar yang diinputkan
  \caption{\emph{Confusion Matrix} Model Hasil Pelatihan}
  % Label referensi dari gambar yang diinputkan
  \label{fig:matrix5}
\end{figure}

Dari analisis tabel performa klasifikasi model dengan jarak 70 sentimeter dapat dilihat untuk kelas Kanan, Kiri, dan Stop, model mencapai \emph{True Positives} (TP) maksimal yaitu 120, dengan \emph{True Negatives} (TN) 480, dan tidak ada \emph{False Positives} (FP) atau \emph{False Negatives} (FN), menandakan tidak ada kesalahan klasifikasi. Sementara itu, kelas Maju juga menunjukkan kinerja yang baik dengan 117 TP dan 3 FN, yang berarti ada tiga kasus di mana gestur Maju tidak diidentifikasi dengan benar. Untuk kelas Mundur, terdapat sedikit gangguan di mana meskipun mendapat 120 TP, terdapat 3 FP yang berarti model salah mengklasifikasikan tiga gestur lain sebagai Mundur.

%Tabel 4.8
\begin{longtable}{|l|c|c|c|c|}
  \caption{Hasil klasifikasi model dengan \emph{confusion matrix}}
  \label{tb:cm_model5} \\
  \hline
  \rowcolor[HTML]{C0C0C0} 
  \textbf{Kelas} & \textbf{TP} & \textbf{TN} & \textbf{FP} & \textbf{FN} \\ \hline
  Kanan    & 120          & 480         & 0           & 0           \\ \hline
  Kiri      & 120          & 480         & 0           & 0           \\ \hline
  Maju      & 117          & 480         & 0           & 3           \\ \hline
  Mundur     & 120          & 477         & 3           & 0           \\ \hline
  Stop  & 120          & 480         & 0           & 0           \\ \hline
\end{longtable}

Setelah didapatkan nilai TP, TN, FP, dan FN untuk masing-masing kelas, selanjutnya didapatkan untuk kelas Kanan, Kiri, dan Stop, model mencapai skor sempurna di semua metrik—\emph{Accuracy, Precision, Recall}, dan \emph{F1-Score}—semuanya 100\%. Sedangkan untuk kelas Maju, model ini masih menunjukkan hasil yang sangat baik dengan \emph{Accuracy} sebesar 99.50\%, \emph{Precision} 100\%, \emph{Recall} 97.50\%, dan \emph{F1-Score} 98.73\%. Untuk kelas Mundur, model mencatat \emph{Accuracy} 99.50\%, namun dengan \emph{Precision} sedikit menurun menjadi 97.56\%, yang mencerminkan adanya beberapa kesalahan dalam memprediksi gestur sebagai Mundur. Namun, dengan \emph{Recall} 100\% dan \emph{F1-Score} 98.77\%, dapat dilihat bahwa model tersebut secara konsisten mengidentifikasi semua gestur Mundur yang sebenarnya, menegaskan keandalan dan efektivitasnya dalam aplikasi praktis.

%Tabel 4.9
\begin{longtable}{|l|c|c|c|c|}
  \caption{Hasil validasi nilai model}
  \label{tb:vs_model5} \\
  \hline
  \rowcolor[HTML]{C0C0C0} 
  \textbf{Kelas} & \textbf{Accuracy} & \textbf{Precision} & \textbf{Recall} & \textbf{F1-Score} \\ \hline
  Kanan    & 100\%            & 100\%             & 100\%           & 100\%            \\ \hline
  Kiri     & 100\%          & 100\%           & 100\%           & 100\%           \\ \hline
  Maju      & 99.50\%          & 100\%           & 97.50\%          & 98.73\%          \\ \hline
  Mundur     & 99.50\%            & 97.56\%             & 100\%           & 98.77\%            \\ \hline
  Stop  & 100\%            & 100\%             & 100\%           & 100\%            \\ \hline
\end{longtable}

\subsection{Jarak 90 sentimeter}

Skenario pengujian jarak yang terakhir adalah dengan jarak 90 sentimeter dari kamera ke objek yang dijadikan data citra. Melalui analisis \emph{confusion matrix} pada Gambar \ref{fig:matrix6} untuk kelas Kanan, model berhasil mengidentifikasi 119 dari 120 gestur dengan benar, hanya salah satu gestur diklasifikasikan sebagai Maju. Untuk kelas Kiri, model menunjukkan kelemahan yang lebih signifikan dengan hanya 114 gestur yang diklasifikasikan dengan benar dan 6 gestur yang salah diklasifikasikan sebagai Maju. Hal ini menunjukkan kesulitan model dalam membedakan gestur Kiri dari Maju. Untuk kelas Maju, ada satu gestur yang salah diklasifikasikan sebagai Mundur. Sementara itu, kelas Mundur dan Stop memiliki tingkat akurasi yang tinggi, meskipun kelas Stop mengalami empat gestur yang salah diklasifikasikan sebagai Mundur. Kesalahan ini mungkin disebabkan oleh kemiripan gestur atau penurunan akurasi deteksi karena peningkatan jarak, sebab semakin jauh jarak maka semakin kecil data citra yang didapat.
  
%Gambar 4.8
\begin{figure} [H] \centering
  % Nama dari file gambar yang diinputkan
  \includegraphics[width=0.25\textwidth]{gambar/bab4/model5 (30cm)/110cm/matrix.png}
  % Keterangan gambar yang diinputkan
  \caption{\emph{Confusion Matrix} Model Hasil Pelatihan}
  % Label referensi dari gambar yang diinputkan
  \label{fig:matrix6}
\end{figure}

Dari hasil perhitungan \emph{confusion matrix} didapatkan hasil klasifikasi model. Untuk kelas Kanan, model hampir sempurna dengan 119 TP dan hanya satu FN, menunjukkan efektivitas tinggi dalam mengenali gestur ini. Sementara itu, kelas Kiri mengalami sedikit kesulitan dengan 6 FN, mengindikasikan beberapa kasus di mana gestur Kiri tidak terdeteksi dengan akurat. Untuk kelas Maju, model ini mengalami 11 FP dan satu FN, menandakan bahwa peningkatan kemampuan model dalam membedakan gestur Maju dari yang lain diperlukan. Kelas Mundur menunjukkan performa yang sangat baik dengan tidak adanya FP atau FN. Akhirnya, untuk kelas Stop, ada satu FP dan empat FN, yang menunjukkan bahwa beberapa gestur Stop tidak teridentifikasi sebagai Stop. Hasil ini menyoroti beberapa kekuatan dan kelemahan dalam model yang perlu diperbaiki untuk mencapai akurasi yang lebih tinggi, terutama dalam mengurangi FN dan FP di kelas tertentu.

%Tabel 4.10
\begin{longtable}{|l|c|c|c|c|}
  \caption{Hasil klasifikasi model dengan \emph{confusion matrix}}
  \label{tb:cm_model6} \\
  \hline
  \rowcolor[HTML]{C0C0C0} 
  \textbf{Kelas} & \textbf{TP} & \textbf{TN} & \textbf{FP} & \textbf{FN} \\ \hline
  Kanan    & 119          & 480         & 0           & 1           \\ \hline
  Kiri      & 114          & 480         & 0           & 6           \\ \hline
  Maju      & 119          & 469         & 11           & 1           \\ \hline
  Mundur     & 120          & 480         & 0           & 0           \\ \hline
  Stop  & 116          & 479         & 1           & 4           \\ \hline
\end{longtable}

Dari hasil perhitungan yang dilakukan berdasarkan nilai TP, TN, FP, dan FN, pada Tabel \ref{tb:vs_model6} berisi hasil validasi untuk setiap kelas dalam model klasifikasi gestur yang dikembangkan. Untuk kelas Kanan, model menunjukkan keakuratan tinggi dengan \textit{Accuracy} sebesar 99.83\%, \textit{Precision} 100\%, \textit{Recall} 99.17\%, dan \textit{F1-Score} 99.58\%. Pada kelas Kiri, model mencatatkan \textit{Accuracy} 99\%, \textit{Precision} 100\%, \textit{Recall} 95\%, dan \textit{F1-Score} 97.44\%. Untuk kelas Maju, metrik yang tercatat adalah \textit{Accuracy} 98\%, \textit{Precision} 91.54\%, \textit{Recall} 99.17\%, dan \textit{F1-Score} 95.20\%. Kelas Mundur menunjukkan performa sempurna dengan 100\% di semua metrik, dan kelas Stop memiliki \textit{Accuracy} 99.17\%, \textit{Precision} 99.15\%, \textit{Recall} 96.67\%, serta \textit{F1-Score} 97.89\%. 

%Tabel 4.11
\begin{longtable}{|l|c|c|c|c|}
  \caption{Hasil validasi nilai model}
  \label{tb:vs_model6} \\
  \hline
  \rowcolor[HTML]{C0C0C0} 
  \textbf{Kelas} & \textbf{Accuracy} & \textbf{Precision} & \textbf{Recall} & \textbf{F1-Score} \\ \hline
  Kanan    & 99.83\%            & 100\%             & 99.17\%           & 99.58\%            \\ \hline
  Kiri     & 99\%          & 100\%           & 95\%           & 97.44\%           \\ \hline
  Maju      & 98\%          & 91.54\%           & 99.17\%          & 95.20\%          \\ \hline
  Mundur     & 100\%            & 100\%             & 100\%           & 100\%            \\ \hline
  Stop  & 99.17\%            & 99.15\%             & 96.67\%           & 97.89\%            \\ \hline
\end{longtable}

Pengujian performa model dengan variasi jarak 30 sentimeter, 50 sentimeter, 70 sentimeter, dan 90 sentimeter menunjukkan tren yang jelas bahwa semakin jauh jarak, performa model semakin berkurang. Hal ini dibuktikan dengan penurunan nilai akurasi, presisi, recall, dan f1 score seiring bertambahnya jarak. Pada jarak 30 sentimeter dan 50 sentimeter, model mencapai performa terbaik dengan akurasi yang tinggi, serta nilai presisi, recall, dan f1 score yang konsisten di setiap kelas perintah. Namun, ketika jarak meningkat 70 sentimeter, mulai terlihat penurunan performa, dengan semakin banyak prediksi yang keliru pada matriks kebingungan. Pada jarak 90 sentimeter, penurunan performa menjadi lebih signifikan dalam nilai presisi, recall, dan f1 score terutama pada kelas "Maju" dan kelas "Mundur".

Secara keseluruhan, hasil pengujian ini menegaskan bahwa jarak antara kamera dan pengguna memiliki pengaruh yang signifikan terhadap performa model. Model bekerja paling optimal pada jarak 30 sentimeter, sedangkan jarak yang lebih jauh seperti 70 sentimeter dan 90 sentimeter performa model menjadi kurang optimal. Dengan demikian, pemilihan jarak ideal menjadi penting dalam implementasi sistem kontrol kursi roda berbasis gerakan mata ini, untuk memastikan akurasi dan responsivitas yang optimal bagi pengguna.

\section{Pengujian Performa Model dengan menggunakan Variasi Pencahayaan}

Pada sistem kontrol kursi roda berbasis \emph{gesture} mata, variasi pencahayaan memiliki pengaruh signifikan terhadap performa model deteksi. Model yang diimplementasikan harus mampu beroperasi secara andal dalam berbagai kondisi pencahayaan. Sub-bab ini akan membahas pengujian performa model dengan menggunakan variasi pencahayaan. Tujuan dari pengujian ini adalah untuk mengevaluasi tingkat keberhasilan dan ketidakberhasilan model dalam mendeteksi gerakan mata pada tiga kondisi pencahayaan, seperti pencahayaan terang (131 Lux), normal (77 Lux), dan redup (35 Lux).

Jenis pencahayaan yang diuji adalah \emph{ambient light}, yang mensimulasikan pencahayaan lingkungan sekitar dengan variasi intensitas yang berbeda. \emph{Ambient light} adalah pencahayaan umum yang ada di sekitar lingkungan, yang dapat berasal dari sumber cahaya alami seperti sinar matahari atau buatan seperti lampu. Tingkat keberhasilan dan ketidakberhasilan dihitung dalam persentase, dengan data yang dikumpulkan selama tiga puluh sesi pengujian. Hasil dari analisis ini akan memberikan gambaran mengenai seberapa baik model dapat beradaptasi terhadap variasi pencahayaan yang berbeda.

\subsection{Pengujian pada cahaya 35 Lux}

Pengujian pada kondisi pencahayaan 35 lux bertujuan untuk mengevaluasi performa model dalam mendeteksi gerakan mata pada situasi pencahayaan yang redup. Kondisi ini mensimulasikan lingkungan dengan cahaya minimal yaitu di ruangan yang hanya diterangi lampu yang redup. Dalam kondisi ini, model diuji untuk mengenali dan mengklasifikasikan perintah gerakan mata dengan benar. Pada pengujian ini ditunjukkan kemampuan model dalam mengeksekusi perintah kontrol pada pencahayaan rendah.

%Tabel 4.12
\begin{longtable}{|l|c|c|c|c|}
  \caption{Pengujian Kinerja Model dengan Pencahayaan 35 Lux}
  \label{tb:lux35} \\
  \hline
  \rowcolor[HTML]{C0C0C0} 
  \textbf{Kelas} & \multicolumn{1}{c|}{\textbf{\begin{tabular}[c]{@{}c@{}}Jumlah \\ Keberhasilan\end{tabular}}} & \multicolumn{1}{c|}{\textbf{\begin{tabular}[c]{@{}c@{}}Jumlah \\ Keridakberhasilan\end{tabular}}} & \multicolumn{1}{c|}{\textbf{\begin{tabular}[c]{@{}c@{}}Persentase \\ Keberhasilan\end{tabular}}} & \multicolumn{1}{c|}{\textbf{\begin{tabular}[c]{@{}c@{}}Persentase\\ Ketidakberhasilan\end{tabular}}} \\ \hline
  Kanan  & 30                                                                                  & 0                                                                                        & 100\%                                                                                   & 0\%                                                                                         \\ \hline
  Kiri   & 30                                                                                  & 0                                                                                        & 100\%                                                                                   & 0\%                                                                                         \\ \hline
  Maju   & 27                                                                                  & 3                                                                                        & 90\%                                                                                    & 10\%                                                                                        \\ \hline
  Mundur & 28                                                                                  & 2                                                                                        & 93.33\%                                                                                 & 6.67\%                                                                                      \\ \hline
  Stop   & 30                                                                                  & 0                                                                                        & 100\%                                                                                   & 0\%                                                                                         \\ \hline
\end{longtable}

Berdasarkan hasil pengujian performa model dengan variasi pencahayaan 35 Lux pada Tabel \ref{tb:lux35}, model menunjukkan tingkat keberhasilan yang sempurna pada kelas "Kanan", "Kiri", dan "Stop" dengan masing-masing tingkat keberhasilan 100\% dan tanpa kesalahan deteksi. Namun, pada kelas "Maju", tingkat keberhasilan sebesar 90\% dengan 3 kesalahan deteksi (10\%), dan pada kelas "Mundur", tingkat keberhasilan 93.33\% dengan 2 kesalahan deteksi (6.67\%). Hasil ini menunjukkan bahwa meskipun model umumnya memiliki performa yang baik, akurasi deteksi pada kelas "Maju" dan "Mundur" masih dipengaruhi oleh variasi pencahayaan.

\subsection{Pengujian pada cahaya 77 Lux}

Pada bagian ini, akan dianalisis bagaimana model beroperasi di bawah pencahayaan sebesar 77 Lux. Pencahayaan ini mensimulasikan kondisi ruangan yang normal. Pengujian ini bertujuan untuk memastikan bahwa model dapat mendeteksi dan mengklasifikasikan gerakan mata dengan akurasi tinggi dalam kondisi pencahayaan normal. Hasil pengujian ini akan memberikan gambaran mengenai kemampuan model beradaptasi dengan baik terhadap variasi pencahayaan dalam lingkungan sehari-hari.

%Tabel 4.13
\begin{longtable}{|l|c|c|c|c|}
  \caption{Pengujian Kinerja Model dengan Pencahayaan 77 Lux}
  \label{tb:lux77} \\
  \hline
  \rowcolor[HTML]{C0C0C0} 
  \textbf{Kelas} & \multicolumn{1}{c|}{\textbf{\begin{tabular}[c]{@{}c@{}}Jumlah \\ Keberhasilan\end{tabular}}} & \multicolumn{1}{c|}{\textbf{\begin{tabular}[c]{@{}c@{}}Jumlah \\ Keridakberhasilan\end{tabular}}} & \multicolumn{1}{c|}{\textbf{\begin{tabular}[c]{@{}c@{}}Persentase \\ Keberhasilan\end{tabular}}} & \multicolumn{1}{c|}{\textbf{\begin{tabular}[c]{@{}c@{}}Persentase\\ Ketidakberhasilan\end{tabular}}} \\ \hline
  Kanan  & 30                                                                                  & 0                                                                                        & 100\%                                                                                   & 0\%                                                                                         \\ \hline
  Kiri   & 30                                                                                  & 0                                                                                        & 100\%                                                                                   & 0\%                                                                                         \\ \hline
  Maju   & 29                                                                                  & 1                                                                                        & 96.67\%                                                                                    & 3.33\%                                                                                        \\ \hline
  Mundur & 28                                                                                  & 2                                                                                        & 93.33\%                                                                                 & 6.67\%                                                                                      \\ \hline
  Stop   & 30                                                                                  & 0                                                                                        & 100\%                                                                                   & 0\%                                                                                         \\ \hline
\end{longtable}

Pada Tabel \ref{tb:lux77}, pengujian kinerja model dengan pencahayaan 77 Lux, hasil menunjukkan bahwa model secara umum memiliki tingkat akurasi yang baik. Perintah "Kanan", "Kiri", dan "Stop" mencapai tingkat keberhasilan 100\% tanpa ada kegagalan deteksi. Sementara itu, perintah "Maju" dan "Mundur" masing-masing mencapai tingkat keberhasilan 96,67\% dan 93,33\%, dengan tingkat ketidakberhasilan sebesar 3,33\% dan 6,67\%. Perbedaan dalam tingkat keberhasilan ini menunjukkan bahwa model sedikit kurang optimal dalam mendeteksi perintah "Maju" dan "Mundur" dibandingkan dengan perintah lainnya. Meski demikian, hasil keseluruhan tetap menunjukkan bahwa model memiliki kinerja yang andal dalam lingkungan bercahaya normal.

\subsection{Pengujian pada cahaya 131 Lux}

Pengujian kinerja model pada kondisi pencahayaan 131 Lux dilakukan untuk mengevaluasi kemampuan sistem kontrol kursi roda berbasis gerakan mata dalam lingkungan dengan pencahayaan terang. Pengujian ini bertujuan untuk memastikan bahwa model dapat mendeteksi dan mengklasifikasikan gerakan mata dengan akurasi tinggi dalam kondisi pencahayaan terang. Hasil pengujian ini akan memberikan gambaran mengenai kemampuan model beradaptasi dengan baik terhadap variasi pencahayaan dalam lingkungan sehari-hari yang lebih terang.

%Tabel 4.14
\begin{longtable}{|l|c|c|c|c|}
  \caption{Pengujian Kinerja Model dengan Pencahayaan 131 Lux}
  \label{tb:lux131} \\
  \hline
  \rowcolor[HTML]{C0C0C0} 
  \textbf{Kelas} & \multicolumn{1}{c|}{\textbf{\begin{tabular}[c]{@{}c@{}}Jumlah \\ Keberhasilan\end{tabular}}} & \multicolumn{1}{c|}{\textbf{\begin{tabular}[c]{@{}c@{}}Jumlah \\ Keridakberhasilan\end{tabular}}} & \multicolumn{1}{c|}{\textbf{\begin{tabular}[c]{@{}c@{}}Persentase \\ Keberhasilan\end{tabular}}} & \multicolumn{1}{c|}{\textbf{\begin{tabular}[c]{@{}c@{}}Persentase\\ Ketidakberhasilan\end{tabular}}} \\ \hline
  Kanan          & 30                                                                                           & 0                                                                                                 & 100\%                                                                                            & 0\%                                                                                                  \\ \hline
  Kiri           & 30                                                                                           & 0                                                                                                 & 100\%                                                                                            & 0\%                                                                                                  \\ \hline
  Maju           & 30                                                                                           & 0                                                                                                 & 100\%                                                                                            & 0\%                                                                                                  \\ \hline
  Mundur         & 30                                                                                           & 0                                                                                                 & 100\%                                                                                            & 0\%                                                                                                  \\ \hline
  Stop           & 30                                                                                           & 0                                                                                                 & 100\%                                                                                            & 0\%                                                                                                  \\ \hline
\end{longtable}

Berdasarkan Tabel \ref{tb:lux131} diatas, hasil menunjukkan bahwa model memiliki tingkat akurasi yang sempurna dalam mendeteksi semua perintah gerakan mata. Setiap perintah, termasuk "Kanan", "Kiri", "Maju", "Mundur", dan "Stop", berhasil terdeteksi dengan tingkat keberhasilan 100\% tanpa ada satu pun kesalahan atau kegagalan deteksi. Hasil ini menegaskan bahwa model bekerja secara optimal dalam kondisi pencahayaan terang, menunjukkan bahwa pencahayaan 131 Lux mendukung model untuk memberikan kinerja yang sangat andal dalam mengenali dan mengklasifikasikan \emph{gesture} mata dengan tepat.

\section{Pengujian Performa \emph{Frame Per Second} (FPS) pada Sistem}

Pengujian performa \emph{Frame Per Second} (FPS) bertujuan untuk mengevaluasi kecepatan sistem dalam memproses gerakan mata secara real-time untuk mengontrol kursi roda. FPS merupakan indikator penting dalam menilai kelancaran dan responsivitas sistem, terutama dalam aplikasi yang memerlukan deteksi gerakan dan pengambilan keputusan yang cepat. Pada sub-bagian ini, dilakukan pengujian performa FPS untuk memastikan bahwa sistem dapat mempertahankan tingkat kecepatan pemrosesan yang memadai, sehingga kontrol kursi roda dapat berjalan lancar dan responsif. Pengujian dilakukan sebanyak tiga puluh kali pada setiap kelas.

Hasil pengujian pada Tabel \ref{tb:fpskanankiri} menunjukkan bahwa performa FPS pada laptop secara konsisten lebih tinggi dibandingkan Intel NUC untuk kedua perintah "Kanan" dan "Kiri". Pada perintah "Kanan," FPS rata-rata pada laptop mencapai 12,745, sementara pada Intel NUC hanya mencapai 10,567. Pada perintah "Kiri", FPS rata-rata pada laptop mencapai 12,635, sedangkan pada Intel NUC hanya mencapai 10,617.

Perbedaan kinerja FPS ini menunjukkan bahwa laptop mampu memproses gambar lebih cepat dan stabil dibandingkan Intel NUC. Untuk perintah "Kanan", kisaran FPS pada laptop berada antara 10,389 hingga 13,959, sementara pada Intel NUC antara 7,762 hingga 12,177. Sedangkan pada perintah "Kiri", kisaran FPS pada laptop berada antara 10,617 hingga 14,280, sementara pada Intel NUC antara 8,554 hingga 13,127.

%Tabel 4.15
\begin{longtable}{|c|c|c|c|c|c|c|}
  \caption{Hasil Performa FPS pada Kelas Kanan dan Kiri}
  \label{tb:fpskanankiri} \\
  \cline{1-3} \cline{5-7}
  \rowcolor[HTML]{C0C0C0} 
  \textbf{Kelas} & \textbf{FPS Laptop} & \textbf{FPS NUC} &\cellcolor[HTML]{FFFFFF}  & \textbf{Kelas} & \textbf{FPS Laptop} & \textbf{FPS NUC} \\ \cline{1-3} \cline{5-7} 
  Kanan          & 13.007              & 10.018           &  & Kiri           & 10.819              & 13.127           \\ \cline{1-3} \cline{5-7} 
  Kanan          & 13.293              & 11.369           &  & Kiri           & 12.150              & 10.089           \\ \cline{1-3} \cline{5-7} 
  Kanan          & 10.927              & 10.043           &  & Kiri           & 13.880              & 11.828           \\ \cline{1-3} \cline{5-7} 
  Kanan          & 10.697              & 7.762            &  & Kiri           & 14.280              & 10.205           \\ \cline{1-3} \cline{5-7} 
  Kanan          & 13.125              & 11.295           &  & Kiri           & 10.916              & 9.879            \\ \cline{1-3} \cline{5-7} 
  Kanan          & 13.672              & 12.177           &  & Kiri           & 12.748              & 8.665            \\ \cline{1-3} \cline{5-7} 
  Kanan          & 11.709              & 11.977           &  & Kiri           & 10.617              & 9.901            \\ \cline{1-3} \cline{5-7} 
  Kanan          & 13.803              & 10.231           &  & Kiri           & 13.403              & 11.823           \\ \cline{1-3} \cline{5-7} 
  Kanan          & 12.866              & 10.057           &  & Kiri           & 12.335              & 10.071           \\ \cline{1-3} \cline{5-7} 
  Kanan          & 12.687              & 11.284           &  & Kiri           & 13.354              & 10.061           \\ \cline{1-3} \cline{5-7} 
  Kanan          & 12.236              & 10.222           &  & Kiri           & 12.110              & 10.147           \\ \cline{1-3} \cline{5-7} 
  Kanan          & 13.295              & 10.307           &  & Kiri           & 13.025              & 11.725           \\ \cline{1-3} \cline{5-7} 
  Kanan          & 12.297              & 10.119           &  & Kiri           & 13.230              & 12.093           \\ \cline{1-3} \cline{5-7} 
  Kanan          & 13.139              & 11.097           &  & Kiri           & 14.249              & 8.660            \\ \cline{1-3} \cline{5-7} 
  Kanan          & 13.439              & 9.025            &  & Kiri           & 11.462              & 11.752           \\ \cline{1-3} \cline{5-7} 
  Kanan          & 13.644              & 11.983           &  & Kiri           & 13.121              & 8.554            \\ \cline{1-3} \cline{5-7} 
  Kanan          & 13.380              & 9.974            &  & Kiri           & 12.739              & 10.001           \\ \cline{1-3} \cline{5-7} 
  Kanan          & 13.271              & 11.311           &  & Kiri           & 11.925              & 9.930            \\ \cline{1-3} \cline{5-7} 
  Kanan          & 12.906              & 8.927            &  & Kiri           & 12.713              & 10.125           \\ \cline{1-3} \cline{5-7} 
  Kanan          & 13.689              & 11.751           &  & Kiri           & 13.314              & 10.047           \\ \cline{1-3} \cline{5-7} 
  Kanan          & 12.937              & 11.887           &  & Kiri           & 13.879              & 11.830           \\ \cline{1-3} \cline{5-7} 
  Kanan          & 12.909              & 9.864            &  & Kiri           & 11.505              & 10.171           \\ \cline{1-3} \cline{5-7} 
  Kanan          & 13.348              & 11.985           &  & Kiri           & 13.413              & 9.846            \\ \cline{1-3} \cline{5-7} 
  Kanan          & 13.959              & 10.505           &  & Kiri           & 12.483              & 10.051           \\ \cline{1-3} \cline{5-7} 
  Kanan          & 10.389              & 9.972            &  & Kiri           & 11.470              & 11.838           \\ \cline{1-3} \cline{5-7} 
  Kanan          & 12.073              & 11.274           &  & Kiri           & 12.808              & 10.086           \\ \cline{1-3} \cline{5-7} 
  Kanan          & 11.279              & 10.540           &  & Kiri           & 12.279              & 11.760           \\ \cline{1-3} \cline{5-7} 
  Kanan          & 11.669              & 10.023           &  & Kiri           & 13.388              & 12.355           \\ \cline{1-3} \cline{5-7} 
  Kanan          & 12.948              & 9.763            &  & Kiri           & 12.619              & 9.906            \\ \cline{1-3} \cline{5-7} 
  Kanan          & 13.765              & 10.257           &  & Kiri           & 12.814              & 11.979           \\ \cline{1-3} \cline{5-7} 
\end{longtable}

Selanjuntya, berdasarkan hasil pengujian performa \emph{Frame Per Second} (FPS) dari Tabel \ref{tb:fpsmajumundur}, pada perintah "Maju," FPS rata-rata pada laptop adalah 13,078 dengan kisaran antara 11,098 hingga 14,142, sedangkan pada Intel NUC, FPS rata-rata adalah 10,068 dengan kisaran antara 7,558 hingga 12,068. Sementara itu, pada perintah "Mundur," FPS rata-rata pada laptop mencapai 13,507 dengan kisaran antara 11,100 hingga 14,458, sementara pada Intel NUC, FPS rata-rata adalah 9,734 dengan kisaran antara 6,661 hingga 12,032.

%Tabel 4.16
\begin{longtable}{|c|c|c|c|c|c|c|}
  \caption{Hasil Performa FPS pada Kelas Maju dan Mundur}
  \label{tb:fpsmajumundur} \\ 
  \cline{1-3} \cline{5-7}
  \rowcolor[HTML]{C0C0C0}
  \textbf{Kelas} & \textbf{FPS Laptop} & \textbf{FPS NUC} &\cellcolor[HTML]{FFFFFF}  & \textbf{Kelas} & \textbf{FPS Laptop} & \textbf{FPS NUC} \\ \cline{1-3} \cline{5-7} 
  Maju           & 13.574              & 9.883            &  & Mundur         & 14.144              & 8.695            \\ \cline{1-3} \cline{5-7} 
  Maju           & 13.792              & 10.030           &  & Mundur         & 14.076              & 9.992            \\ \cline{1-3} \cline{5-7} 
  Maju           & 11.709              & 12.068           &  & Mundur         & 13.874              & 9.845            \\ \cline{1-3} \cline{5-7} 
  Maju           & 13.372              & 11.429           &  & Mundur         & 11.190              & 11.983           \\ \cline{1-3} \cline{5-7} 
  Maju           & 13.923              & 9.969            &  & Mundur         & 13.725              & 10.024           \\ \cline{1-3} \cline{5-7} 
  Maju           & 13.048              & 10.374           &  & Mundur         & 13.693              & 11.980           \\ \cline{1-3} \cline{5-7} 
  Maju           & 12.818              & 10.066           &  & Mundur         & 14.066              & 10.045           \\ \cline{1-3} \cline{5-7} 
  Maju           & 11.867              & 11.375           &  & Mundur         & 14.028              & 10.116           \\ \cline{1-3} \cline{5-7} 
  Maju           & 12.701              & 10.430           &  & Mundur         & 13.852              & 9.863            \\ \cline{1-3} \cline{5-7} 
  Maju           & 13.710              & 10.021           &  & Mundur         & 13.271              & 11.958           \\ \cline{1-3} \cline{5-7} 
  Maju           & 13.882              & 9.565            &  & Mundur         & 13.308              & 10.026           \\ \cline{1-3} \cline{5-7} 
  Maju           & 12.945              & 8.903            &  & Mundur         & 14.076              & 10.444           \\ \cline{1-3} \cline{5-7} 
  Maju           & 12.120              & 11.475           &  & Mundur         & 13.201              & 11.415           \\ \cline{1-3} \cline{5-7} 
  Maju           & 13.302              & 8.879            &  & Mundur         & 14.040              & 7.512            \\ \cline{1-3} \cline{5-7} 
  Maju           & 12.778              & 10.026           &  & Mundur         & 13.606              & 9.935            \\ \cline{1-3} \cline{5-7} 
  Maju           & 13.728              & 8.557            &  & Mundur         & 14.195              & 8.612            \\ \cline{1-3} \cline{5-7} 
  Maju           & 13.972              & 10.014           &  & Mundur         & 13.660              & 8.403            \\ \cline{1-3} \cline{5-7} 
  Maju           & 13.437              & 9.865            &  & Mundur         & 13.265              & 10.269           \\ \cline{1-3} \cline{5-7} 
  Maju           & 12.809              & 10.123           &  & Mundur         & 14.458              & 8.545            \\ \cline{1-3} \cline{5-7} 
  Maju           & 12.971              & 10.102           &  & Mundur         & 13.465              & 10.016           \\ \cline{1-3} \cline{5-7} 
  Maju           & 12.206              & 9.941            &  & Mundur         & 13.505              & 9.981            \\ \cline{1-3} \cline{5-7} 
  Maju           & 13.330              & 7.558            &  & Mundur         & 13.840              & 9.979            \\ \cline{1-3} \cline{5-7} 
  Maju           & 13.263              & 9.849            &  & Mundur         & 13.855              & 10.015           \\ \cline{1-3} \cline{5-7} 
  Maju           & 14.142              & 11.985           &  & Mundur         & 13.223              & 12.032           \\ \cline{1-3} \cline{5-7} 
  Maju           & 11.098              & 9.988            &  & Mundur         & 14.083              & 7.481            \\ \cline{1-3} \cline{5-7} 
  Maju           & 12.809              & 9.782            &  & Mundur         & 13.281              & 6.661            \\ \cline{1-3} \cline{5-7} 
  Maju           & 12.813              & 8.755            &  & Mundur         & 11.100              & 8.592            \\ \cline{1-3} \cline{5-7} 
  Maju           & 13.392              & 9.944            &  & Mundur         & 13.888              & 7.570            \\ \cline{1-3} \cline{5-7} 
  Maju           & 13.915              & 10.013           &  & Mundur         & 12.248              & 10.004           \\ \cline{1-3} \cline{5-7} 
  Maju           & 12.905              & 11.077           &  & Mundur         & 12.984              & 10.034           \\ \cline{1-3} \cline{5-7} 
\end{longtable}

Pada hasil pengujian terakhir di Tabel \ref{tb:fpsstop} ditunjukkan bahwa pada laptop, kisaran FPS menunjukkan stabilitas yang cukup baik, dengan nilai minimum 10,415 dan nilai maksimum 13,972. Sebaliknya, kisaran FPS pada Intel NUC lebih variatif, dengan nilai minimum serendah 8,405 dan nilai maksimum 12,448.

%Tabel 4.17
\begin{longtable}{|c|c|c|}
  \caption{Hasil Performa FPS pada Kelas Stop}
  \label{tb:fpsstop} \\
  \hline
  \rowcolor[HTML]{C0C0C0} 
  \textbf{Kelas} & \textbf{FPS Laptop} & \textbf{FPS NUC} \\ \hline
  Stop           & 13.574              & 10.009           \\ \hline
  Stop           & 13.792              & 8.586            \\ \hline
  Stop           & 11.709              & 11.490           \\ \hline
  Stop           & 13.372              & 10.355           \\ \hline
  Stop           & 13.923              & 8.607            \\ \hline
  Stop           & 10.987              & 9.676            \\ \hline
  Stop           & 10.857              & 12.448           \\ \hline
  Stop           & 12.701              & 9.958            \\ \hline
  Stop           & 10.875              & 8.405            \\ \hline
  Stop           & 10.787              & 10.278           \\ \hline
  Stop           & 12.945              & 11.967           \\ \hline
  Stop           & 12.120              & 8.564            \\ \hline
  Stop           & 13.302              & 8.573            \\ \hline
  Stop           & 12.778              & 10.062           \\ \hline
  Stop           & 13.728              & 9.948            \\ \hline
  Stop           & 13.972              & 10.009           \\ \hline
  Stop           & 13.437              & 10.122           \\ \hline
  Stop           & 12.809              & 11.799           \\ \hline
  Stop           & 12.971              & 12.034           \\ \hline
  Stop           & 12.206              & 10.029           \\ \hline
  Stop           & 10.415              & 9.972            \\ \hline
  Stop           & 13.263              & 9.994            \\ \hline
  Stop           & 11.098              & 10.029           \\ \hline
  Stop           & 10.440              & 9.972            \\ \hline
  Stop           & 13.392              & 11.520           \\ \hline
  Stop           & 13.915              & 8.831            \\ \hline
  Stop           & 12.905              & 8.593            \\ \hline
  Stop           & 12.813              & 11.931           \\ \hline
  Stop           & 12.778              & 10.010           \\ \hline
  Stop           & 13.530              & 10.028           \\ \hline
\end{longtable}

Secara keseluruhan, laptop memiliki performa FPS rata-rata yang lebih tinggi dibandingkan Intel NUC untuk semua perintah gerakan mata. Laptop mampu memproses gerakan mata dengan kecepatan dan stabilitas yang lebih baik, memberikan kontrol yang lebih responsif terhadap kursi roda. Variabilitas FPS pada laptop juga lebih rendah, menunjukkan kestabilan pemrosesan yang lebih baik dibandingkan Intel NUC.

Meskipun performa FPS pada Intel NUC lebih rendah dibandingkan laptop, perangkat ini tetap memiliki beberapa keunggulan. Pertama, Intel NUC memiliki desain yang kecil dan kompak, sehingga lebih mudah dipasang pada kursi roda dan tidak memakan banyak ruang. Kedua, konsumsi daya Intel NUC lebih rendah dibandingkan laptop, membuatnya lebih efisien dan ideal untuk aplikasi dengan daya terbatas. Ketiga, Intel NUC menyediakan berbagai opsi konektivitas, seperti HDMI, USB, dan jaringan, sehingga memudahkan integrasi dengan komponen sistem lain. Terakhir, dibandingkan dengan laptop, Intel NUC cenderung lebih terjangkau, sehingga dapat mengurangi biaya total sistem.

\section{Pengujian Waktu dari \emph{Inference Time} pada Model ke \emph{Response Time} pada Motor Kursi Roda}

Pengujian waktu dari \emph{Inference Time} pada model ke \emph{Response Time} pada motor kursi roda bertujuan untuk mengevaluasi seberapa cepat sistem kontrol kursi roda berbasis gerakan mata dapat merespons perintah dari pengguna. \emph{Inference Time} mengukur waktu yang dibutuhkan oleh model untuk mendeteksi dan mengklasifikasikan gerakan mata, sementara \emph{Response Time} mengukur waktu yang diperlukan motor kursi roda untuk mulai bergerak setelah menerima sinyal dari model. \emph{Response Time} dihitung dengan mengurangi waktu antara \emph{Motor Time} dengan \emph{Sent Time}. Pengujian ini sangat penting karena waktu respons yang cepat dan akurat merupakan kunci dalam memberikan kontrol yang aman dan nyaman bagi pengguna kursi roda. Untuk mendapatkan hasil yang lebih representatif, pengujian akan dilakukan untuk setiap kelas perintah (seperti "Maju," "Mundur," "Kanan," "Kiri," dan "Stop") sebanyak 30 kali pengujian per kelas.

Pertama-tama dilakukan pengujian waktu dari \emph{Inference Time} pada model ke \emph{Response Time} pada motor kursi roda untuk perintah "Kanan." Pengujian ini bertujuan untuk mengevaluasi seberapa cepat sistem kontrol kursi roda berbasis gerakan mata dapat merespons perintah "Kanan" dari pengguna. Setiap pengujian dilakukan sebanyak 30 kali untuk memastikan hasil yang konsisten dan representatif. 

Berdasarkan Tabel \ref{tb:delaykanan} di bawah, ditunjukkan waktu dari \emph{Inference Time} pada model ke \emph{Response Time} pada motor kursi roda untuk perintah "Kanan." Rata-rata \emph{response time} keseluruhan untuk perintah "Kanan" adalah 0,2328 detik, dengan variabilitas waktu respons berkisar antara 0,035 hingga 0,399 detik. \emph{Inference Time}, yaitu waktu yang dibutuhkan oleh model untuk mendeteksi dan mengklasifikasikan perintah "Kanan," memiliki rata-rata sekitar 0,0626 detik. Variabilitas \emph{Inference Time} cukup kecil, dengan kisaran antara 0,0491 hingga 0,0777 detik. Hasil ini menunjukkan bahwa sistem dapat merespons perintah "Kanan" dengan cepat dan akurat.

%Tabel 4.18
\begin{longtable}{|c|c|c|c|c|c|}
  \caption{Hasil Pengujian \emph{Response Time} pada Kelas Kanan}
  \label{tb:delaykanan} \\
  \hline
  \rowcolor[HTML]{C0C0C0} 
  \multicolumn{1}{|l|}{\textbf{Kelas}} & \multicolumn{1}{c|}{\textbf{\begin{tabular}[c]{@{}c@{}}Inference \\ Time\end{tabular}}} & \multicolumn{1}{c|}{\textbf{Sent Time}} & \multicolumn{1}{l|}{\textbf{Received Time}} & \multicolumn{1}{l|}{\textbf{Motor Time}} & \multicolumn{1}{c|}{\textbf{\begin{tabular}[c]{@{}c@{}}Response \\ Time\end{tabular}}} \\ \hline
      Kanan & 0.062386 & 21:48:29.354 & 21:48:29.394 & 21:48:29.394 & 0.040 \\ \hline
      Kanan & 0.061566 & 21:48:29.735 & 21:48:29.780 & 21:48:29.780 & 0.045 \\ \hline
      Kanan & 0.077753 & 21:48:31.928 & 21:48:32.256 & 21:48:32.256 & 0.328 \\ \hline
      Kanan & 0.074318 & 21:48:37.669 & 21:48:37.748 & 21:48:37.748 & 0.079 \\ \hline
      Kanan & 0.064420 & 21:48:39.511 & 21:48:39.716 & 21:48:39.909 & 0.398 \\ \hline
      Kanan & 0.049143 & 21:48:47.362 & 21:48:47.397 & 21:48:47.596 & 0.234 \\ \hline
      Kanan & 0.061485 & 21:48:48.393 & 21:48:48.430 & 21:48:48.616 & 0.223 \\ \hline
      Kanan & 0.062637 & 21:48:50.316 & 21:48:50.664 & 21:48:50.664 & 0.348 \\ \hline
      Kanan & 0.063710 & 21:48:53.683 & 21:48:53.734 & 21:48:53.734 & 0.051 \\ \hline
      Kanan & 0.064039 & 21:48:59.258 & 21:48:59.318 & 21:48:59.495 & 0.237 \\ \hline
      Kanan & 0.064898 & 21:49:00.565 & 21:49:00.601 & 21:49:00.819 & 0.254 \\ \hline
      Kanan & 0.076225 & 21:49:03.664 & 21:49:03.713 & 21:49:03.924 & 0.260 \\ \hline
      Kanan & 0.059118 & 21:49:07.200 & 21:49:07.281 & 21:49:07.484 & 0.284 \\ \hline
      Kanan & 0.064808 & 21:49:10.358 & 21:49:10.415 & 21:49:10.639 & 0.281 \\ \hline
      Kanan & 0.062542 & 21:49:12.435 & 21:49:12.472 & 21:49:12.689 & 0.254 \\ \hline
      Kanan & 0.052755 & 21:49:16.864 & 21:49:16.917 & 21:49:17.143 & 0.279 \\ \hline
      Kanan & 0.064110 & 21:49:19.212 & 21:49:19.251 & 21:49:19.447 & 0.235 \\ \hline
      Kanan & 0.062850 & 21:49:19.631 & 21:49:19.780 & 21:49:19.957 & 0.326 \\ \hline
      Kanan & 0.072371  & 21:49:20.451 & 21:49:20.502 & 21:49:20.696 & 0.245 \\ \hline
      Kanan & 0.062699 & 21:49:21.220 & 21:49:21.265 & 21:49:21.451 & 0.231 \\ \hline
      Kanan & 0.059212 & 21:49:24.901 & 21:49:24.948 & 21:49:25.127 & 0.226 \\ \hline
      Kanan & 0.061927 & 21:49:26.274 & 21:49:26.332 & 21:49:26.332 & 0.058 \\ \hline
      Kanan & 0.058036 & 21:49:27.215 & 21:49:27.250 & 21:49:27.250 & 0.035 \\ \hline
      Kanan & 0.061857 & 21:49:28.731 & 21:49:28.890 & 21:49:29.108 & 0.377 \\ \hline
      Kanan & 0.064971 & 21:49:29.481 & 21:49:29.675 & 21:49:29.862 & 0.381 \\ \hline
      Kanan & 0.061562 & 21:49:32.897 & 21:49:33.046 & 21:49:33.232 & 0.335 \\ \hline
      Kanan & 0.063810 & 21:49:34.160 & 21:49:34.216 & 21:49:34.392 & 0.232 \\ \hline
      Kanan & 0.064849 & 21:49:34.481 & 21:49:34.691 & 21:49:34.880 & 0.399 \\ \hline
      Kanan & 0.076901 & 21:49:36.770 & 21:49:36.815 & 21:49:37.018 & 0.248 \\ \hline
      Kanan & 0.073888 & 21:49:37.936 & 21:49:37.998 & 21:49:37.998 & 0.062 \\ \hline
\end{longtable}

Selanjutnya, pada Tabel \ref{tb:delaykiri} ditunjukkan waktu dari \emph{Inference Time} pada model ke \emph{Response Time} pada motor kursi roda untuk perintah "Kiri." Rata-rata waktu respons keseluruhan untuk perintah "Kiri" adalah 0,0933 detik, dengan variabilitas yang relatif rendah di antara 30 pengujian. Waktu rata-rata untuk deteksi dan klasifikasi perintah "Kiri" oleh model (\emph{Inference Time}) adalah sekitar 0,0640 detik, dengan kisaran antara 0,0525 hingga 0,0826 detik.

Sementara itu, rata-rata waktu respons motor kursi roda (\emph{Response Time}) untuk perintah "Kiri" adalah 0,0933 detik, dengan variabilitas waktu respons berkisar antara 0,023 hingga 0,234 detik. Waktu pengiriman perintah (\emph{Sent Time}) dan penerimaan sinyal (\emph{Received Time}) menunjukkan keterlambatan minimal antara model dan motor, dengan selisih biasanya dalam hitungan milidetik. Hasil ini menunjukkan bahwa sistem dapat merespons perintah "Kiri" dengan cepat dan akurat.

%Tabel 4.19
\begin{longtable}{|c|c|c|c|c|c|}
  \caption{Hasil Pengujian \emph{Response Time} pada Kelas Kiri}
  \label{tb:delaykiri} \\
  \hline
  \rowcolor[HTML]{C0C0C0}
  \multicolumn{1}{|l|}{\textbf{Kelas}} & \multicolumn{1}{c|}{\textbf{\begin{tabular}[c]{@{}c@{}}Inference \\ Time\end{tabular}}} & \multicolumn{1}{c|}{\textbf{Sent Time}} & \multicolumn{1}{l|}{\textbf{Received Time}} & \multicolumn{1}{l|}{\textbf{Motor Time}} & \multicolumn{1}{c|}{\textbf{\begin{tabular}[c]{@{}c@{}}Response \\ Time\end{tabular}}} \\ \hline
      Kiri & 0.052666 & 21:14:16.832 & 21:14:16.882 & 21:14:16.882 & 0.050 \\ \hline
      Kiri & 0.061696 & 21:14:20.393 & 21:14:20.446 & 21:14:20.446 & 0.053 \\ \hline
      Kiri & 0.052505 & 21:14:21.589 & 21:14:21.712 & 21:14:21.712 & 0.123 \\ \hline
      Kiri & 0.064479 & 21:14:24.341 & 21:14:24.398 & 21:14:24.398 & 0.057 \\ \hline
      Kiri & 0.060470 & 21:14:24.813 & 21:14:24.866 & 21:14:24.911 & 0.098 \\ \hline
      Kiri & 0.068255 & 21:14:26.515 & 21:14:26.565 & 21:14:26.565 & 0.050 \\ \hline
      Kiri & 0.066894 & 21:14:26.881 & 21:14:26.933 & 21:14:26.933 & 0.052 \\ \hline
      Kiri & 0.059064 & 21:14:28.881 & 21:14:28.963 & 21:14:28.963 & 0.082 \\ \hline
      Kiri & 0.065253 & 21:14:29.129 & 21:14:29.162 & 21:14:29.162 & 0.033 \\ \hline
      Kiri & 0.060472 & 21:14:29.942 & 21:14:29.996 & 21:14:29.996 & 0.054 \\ \hline
      Kiri & 0.067407 & 21:14:44.562 & 21:14:44.585 & 21:14:44.585 & 0.023 \\ \hline
      Kiri & 0.065442 & 21:14:44.713 & 21:14:44.746 & 21:14:44.746 & 0.033 \\ \hline
      Kiri & 0.063255 & 21:14:53.652 & 21:14:53.696 & 21:14:53.696 & 0.044 \\ \hline
      Kiri & 0.064828 & 21:14:56.630 & 21:14:56.665 & 21:14:56.880  & 0.250 \\ \hline
      Kiri & 0.061501 & 21:15:00.255 & 21:15:00.297 & 21:15:00.297 & 0.042 \\ \hline
      Kiri & 0.082621  & 21:15:09.797 & 21:15:09.866 & 21:15:09.866 & 0.069 \\ \hline
      Kiri & 0.079509 & 21:15:09.982 & 21:15:10.032 & 21:15:10.032 & 0.050 \\ \hline
      Kiri & 0.069636 & 21:15:10.308 & 21:15:10.364 & 21:15:10.364 & 0.056 \\ \hline
      Kiri & 0.068341 & 21:15:10.632 & 21:15:10.696 & 21:15:10.696 & 0.064 \\ \hline
      Kiri & 0.065245 & 21:15:12.648 & 21:15:12.696 & 21:15:12.696 & 0.048 \\ \hline
      Kiri & 0.054735 & 21:15:13.466 & 21:15:13.531 & 21:15:13.531 & 0.065 \\ \hline
      Kiri & 0.064253 & 21:15:16.249 & 21:15:16.313 & 21:15:16.489 & 0.240 \\ \hline
      Kiri & 0.065417 & 21:15:17.345 & 21:15:17.398 & 21:15:17.579 & 0.234 \\ \hline
      Kiri & 0.068259 & 21:15:18.662 & 21:15:18.796 & 21:15:18.796 & 0.134 \\ \hline
      Kiri & 0.061768 & 21:32:33.241 & 21:32:33.275 & 21:32:33.460 & 0.219 \\ \hline
      Kiri & 0.069039 & 21:32:51.239 & 21:32:51.299 & 21:32:51.299 & 0.060 \\ \hline
      Kiri & 0.055086 & 21:32:53.447 & 21:32:53.633 & 21:32:53.633 & 0.186 \\ \hline
      Kiri & 0.063465 & 21:32:57.900 & 21:32:57.949 & 21:32:57.949 & 0.049 \\ \hline
      Kiri & 0.072116 & 21:33:01.949 & 21:33:02.177 & 21:33:02.177 & 0.228 \\ \hline
      Kiri & 0.063332 & 21:33:03.528 & 21:33:03.580 & 21:33:03.580 & 0.052 \\ \hline
\end{longtable}

Pada hasil pengujian berikutnya yang tertera pada Tabel \ref{tb:delaymaju} di bawah, ditunjukkan waktu dari \emph{Inference Time} pada model ke \emph{Response Time} pada motor kursi roda untuk perintah "Maju." Rata-rata waktu respons keseluruhan untuk perintah "Maju" adalah 0,4337 detik, dengan variabilitas yang relatif tinggi di antara 30 pengujian. Waktu rata-rata untuk deteksi dan klasifikasi perintah "Maju" oleh model (\emph{Inference Time}) adalah sekitar 0,0656 detik, dengan variabilitas yang kecil, kisaran antara 0,0560 hingga 0,0753 detik.

Rata-rata waktu respons motor kursi roda (\emph{Response Time}) adalah 0,4337 detik, dengan variabilitas waktu respons yang lebih besar, berkisar antara 0,217 hingga 0,767 detik. Waktu pengiriman perintah (\emph{Sent Time}) dan penerimaan sinyal (\emph{Received Time}) menunjukkan keterlambatan minimal antara model dan motor, dengan selisih biasanya hanya dalam hitungan milidetik. Variabilitas waktu respons cukup tinggi, yang berarti konsistensi respons motor untuk perintah "Maju" cukup baik. Secara keseluruhan, hasil ini menunjukkan bahwa sistem dapat merespons perintah "Maju" dengan cepat.

%Tabel 4.20
\begin{longtable}{|c|c|c|c|c|c|c|c|}
  \caption{Hasil Pengujian \emph{Response Time} pada Kelas Maju}
  \label{tb:delaymaju} \\
  \hline
  \rowcolor[HTML]{C0C0C0} 
  \multicolumn{1}{|l|}{\textbf{Kelas}} & \multicolumn{1}{c|}{\textbf{\begin{tabular}[c]{@{}c@{}}Inference \\ Time\end{tabular}}} & \multicolumn{1}{c|}{\textbf{Sent Time}} & \multicolumn{1}{l|}{\textbf{Received Time}} & \multicolumn{1}{l|}{\textbf{Motor Time}} & \multicolumn{1}{c|}{\textbf{\begin{tabular}[c]{@{}c@{}}Response \\ Time\end{tabular}}} \\ \hline
      Maju & 0.071632 & 02:19:50.650 & 02:19:50.687 & 02:19:51.071 & 0.421 \\ \hline
      Maju & 0.072398 & 02:19:52.771 & 02:19:52.839 & 02:19:52.988 & 0.217 \\ \hline
      Maju & 0.060674 & 02:19:55.208 & 02:19:55.257 & 02:19:55.618 & 0.410 \\ \hline
      Maju & 0.064356 & 02:19:58.239 & 02:19:58.272 & 02:19:58.638 & 0.399 \\ \hline
      Maju & 0.070158 & 02:20:00.356 & 02:20:00.389 & 02:20:00.772 & 0.416 \\ \hline
      Maju & 0.067166 & 02:20:03.256 & 02:20:03.376 & 02:20:03.739 & 0.483 \\ \hline
      Maju & 0.063778 & 02:20:06.022 & 02:20:06.240 & 02:20:06.405 & 0.383 \\ \hline
      Maju & 0.066648 & 02:20:08.639 & 02:20:08.687 & 02:20:09.055  & 0.416 \\ \hline
      Maju & 0.075270 & 02:34:14.568 & 02:34:14.671 & 02:34:15.037 & 0.469 \\ \hline
      Maju & 0.056018 & 02:34:16.524 & 02:34:16.872  & 02:34:17.221 & 0.697 \\ \hline
      Maju & 0.070915 & 02:34:18.773 & 02:34:18.821 & 02:34:19.188 & 0.415 \\ \hline
      Maju & 0.071782 & 02:34:19.323 & 02:34:19.732 & 02:34:20.090 & 0.767 \\ \hline
      Maju & 0.065364 & 02:34:24.173 & 02:34:24.224  & 02:34:24.589  & 0.416 \\ \hline
      Maju & 0.071346 & 02:34:27.122 & 02:34:27.172 & 02:34:27.569 & 0.447 \\ \hline
      Maju & 0.063106 & 02:34:29.975 & 02:34:30.006 & 02:34:30.375 & 0.400 \\ \hline
      Maju & 0.062175 & 02:34:32.807 & 02:34:32.854 & 02:34:33.202 & 0.395 \\ \hline
      Maju & 0.069827 & 02:34:35.924 & 02:34:36.005 & 02:34:36.356 & 0.432 \\ \hline
      Maju & 0.072240 & 02:34:38.693 & 02:34:38.771 & 02:34:39.107 & 0.414 \\ \hline
      Maju & 0.070818 & 02:34:41.712 & 02:34:41.873 & 02:34:42.255 & 0.543 \\ \hline
      Maju & 0.070529 & 02:34:45.056 & 02:34:45.088 & 02:34:45.458 & 0.402 \\ \hline
      Maju & 0.064209 & 02:34:49.155 & 02:34:49.188 & 02:34:49.555 & 0.400 \\ \hline
      Maju & 0.061892 & 03:01:40.959 & 03:01:41.007 & 03:01:41.193 & 0.234 \\ \hline
      Maju & 0.071453 & 03:01:43.308 & 03:01:43.376  & 03:01:43.743 & 0.435 \\ \hline
      Maju & 0.059348 & 03:01:45.258 & 03:01:45.305 & 03:01:45.672 & 0.414 \\ \hline
      Maju & 0.058776 & 03:01:46.727 & 03:01:46.789  & 03:01:47.191 & 0.464 \\ \hline
      Maju & 0.065642 & 03:01:49.109 & 03:01:49.158 & 03:01:49.526 & 0.417 \\ \hline
      Maju & 0.060525 & 03:01:50.560 & 03:01:50.605 & 03:01:51.005 & 0.445 \\ \hline
      Maju & 0.065441 & 03:01:51.927 & 03:01:51.959  & 03:01:52.325 & 0.398 \\ \hline
      Maju & 0.062838 & 03:01:53.258 & 03:01:53.310 & 03:01:53.673 & 0.415 \\ \hline
      Maju & 0.062296 & 03:01:55.326 & 03:01:55.375 & 03:01:55.772 & 0.446 \\ \hline
\end{longtable}

Hasil pengujian berikutnya dapat dilihat pada Tabel \ref{tb:delaymundur} di bawah menunjukkan waktu dari \emph{Inference Time} pada model ke \emph{Response Time} pada motor kursi roda untuk perintah "Mundur." Rata-rata waktu respons keseluruhan untuk perintah "Mundur" adalah 0,1409 detik, dengan variabilitas yang relatif rendah di antara 30 pengujian. Waktu rata-rata untuk deteksi dan klasifikasi perintah "Mundur" oleh model (\emph{Inference Time}) adalah sekitar 0,0637 detik, dengan variabilitas yang kecil, kisaran antara 0,0481 hingga 0,0795 detik.

Rata-rata waktu respons motor kursi roda (\emph{Response Time}) adalah 0,1409 detik, dengan variabilitas waktu respons berkisar antara 0,026 hingga 0,426 detik.Meskipun variabilitas waktu respons masih ada, hasil ini menunjukkan bahwa sistem kontrol kursi roda berbasis gerakan mata mampu memberikan kontrol yang andal dan responsif terhadap perintah "Mundur."

%Tabel 4.21
\begin{longtable}{|c|c|c|c|c|c|}
  \caption{Hasil Pengujian \emph{Response Time} pada Kelas Mundur}
  \label{tb:delaymundur} \\
  \hline
  \rowcolor[HTML]{C0C0C0} 
  \multicolumn{1}{|l|}{\textbf{Kelas}} & \multicolumn{1}{c|}{\textbf{\begin{tabular}[c]{@{}c@{}}Inference \\ Time\end{tabular}}} & \multicolumn{1}{c|}{\textbf{Sent Time}} & \multicolumn{1}{l|}{\textbf{Received Time}} & \multicolumn{1}{l|}{\textbf{Motor Time}} & \multicolumn{1}{c|}{\textbf{\begin{tabular}[c]{@{}c@{}}Response \\ Time\end{tabular}}} \\ \hline
      Mundur & 0.071819 & 03:31:55.289 & 03:31:55.373 & 03:31:55.540 & 0.251 \\ \hline
      Mundur & 0.064442 & 03:32:00.337 & 03:32:00.374 & 03:32:00.408 & 0.071 \\ \hline
      Mundur & 0.079433 & 03:32:00.691 & 03:32:00.742 & 03:32:00.742 & 0.051 \\ \hline
      Mundur & 0.057472 & 03:32:02.606 & 03:32:02.640 & 03:32:02.838 & 0.232 \\ \hline
      Mundur & 0.063444 & 03:32:10.982 & 03:32:11.024 & 03:32:11.250 & 0.268 \\ \hline
      Mundur & 0.058818 & 03:32:18.480 & 03:32:18.691 & 03:32:18.906 & 0.426 \\ \hline
      Mundur & 0.058210 & 03:32:22.986 & 03:32:23.159  & 03:32:23.344 & 0.358 \\ \hline
      Mundur & 0.057689 & 03:32:26.517 & 03:32:26.710 & 03:32:26.888 & 0.371 \\ \hline
      Mundur & 0.062615 & 03:32:29.801 & 03:32:29.831 & 03:32:30.048 & 0.247 \\ \hline
      Mundur & 0.060700 & 03:32:30.523 & 03:32:30.705 & 03:32:30.705 & 0.182 \\ \hline
      Mundur & 0.060621 & 03:32:31.425 & 03:32:31.460 & 03:32:31.460 & 0.035 \\ \hline
      Mundur & 0.056186 & 03:32:31.878 & 03:32:31.904 & 03:32:31.904 & 0.026 \\ \hline
      Mundur & 0.061846 & 03:32:32.357 & 03:32:32.407  & 03:32:32.407  & 0.050 \\ \hline
      Mundur & 0.065102 & 03:32:32.715 & 03:32:32.773 & 03:32:32.773 & 0.058 \\ \hline
      Mundur & 0.076641 & 03:32:35.675 & 03:32:35.708 & 03:32:35.742 & 0.067 \\ \hline
      Mundur & 0.072828 & 03:32:36.859 & 03:32:36.891 & 03:32:36.891 & 0.032 \\ \hline
      Mundur & 0.063810 & 03:32:38.480 & 03:32:38.524 & 03:32:38.724 & 0.244 \\ \hline
      Mundur & 0.061743 & 03:32:38.891 & 03:32:38.941 & 03:32:38.941 & 0.050 \\ \hline
      Mundur & 0.076373 & 03:32:40.707 & 03:32:40.745 & 03:32:40.745 & 0.038 \\ \hline
      Mundur & 0.060289 & 03:32:40.875 & 03:32:40.907 & 03:32:40.907 & 0.032 \\ \hline
      Mundur & 0.063358 & 03:32:43.505 & 03:32:43.590 & 03:32:43.768 & 0.263 \\ \hline
      Mundur & 0.069278 & 03:32:47.542 & 03:32:47.576 & 03:32:47.576 & 0.034 \\ \hline
      Mundur & 0.059765 & 03:32:47.707 & 03:32:47.739  & 03:32:47.739  & 0.032 \\ \hline
      Mundur & 0.048065 & 03:32:48.408 & 03:32:48.457 & 03:32:48.457 & 0.049 \\ \hline
      Mundur & 0.079513 & 03:32:49.566 & 03:32:49.673 & 03:32:49.673 & 0.107 \\ \hline
      Mundur & 0.066401 & 03:32:50.537 & 03:32:50.581 & 03:32:50.581 & 0.044 \\ \hline
      Mundur & 0.062315 & 03:32:51.670 & 03:32:51.723 & 03:32:51.723 & 0.053 \\ \hline
      Mundur & 0.065300 & 03:32:52.614 & 03:32:52.643 & 03:32:52.881 & 0.267 \\ \hline
      Mundur & 0.058984 & 03:32:58.671 & 03:32:58.724  & 03:32:58.906 & 0.235 \\ \hline
      Mundur & 0.065692 & 03:33:00.552 & 03:33:00.607 & 03:33:00.607 & 0.055 \\ \hline
\end{longtable}

Pengujian yang terakhir dapat dilihat pada Tabel \ref{tb:delaystop} di bawah yang menunjukkan waktu dari \emph{Inference Time} pada model ke \emph{Response Time} pada motor kursi roda untuk perintah "Stop." Rata-rata waktu respons keseluruhan untuk perintah "Stop" adalah 0,4318 detik, dengan variabilitas yang relatif rendah di antara 30 pengujian. Waktu rata-rata untuk deteksi dan klasifikasi perintah "Stop" oleh model (\emph{Inference Time}) adalah sekitar 0,0666 detik, dengan variabilitas yang kecil, kisaran antara 0,0554 hingga 0,0816 detik.

Rata-rata waktu respons motor kursi roda (\emph{Response Time}) adalah 0,4318 detik, dengan variabilitas waktu respons berkisar antara 0,396 hingga 0,785 detik. Waktu pengiriman perintah (\emph{Sent Time}) dan penerimaan sinyal (\emph{Received Time}) menunjukkan keterlambatan minimal antara model dan motor, dengan selisih hanya dalam hitungan milidetik. Hasil ini menunjukkan bahwa sistem kontrol kursi roda berbasis gerakan mata mampu memberikan kontrol yang andal dan responsif terhadap perintah "Stop."

%Tabel 4.22
\begin{longtable}{|c|c|c|c|c|c|}
  \caption{Hasil Pengujian \emph{Response Time} pada Kelas Stop}
  \label{tb:delaystop} \\
  \hline
  \rowcolor[HTML]{C0C0C0} 
  \multicolumn{1}{|l|}{\textbf{Kelas}} & \multicolumn{1}{c|}{\textbf{\begin{tabular}[c]{@{}c@{}}Inference \\ Time\end{tabular}}} & \multicolumn{1}{c|}{\textbf{Sent Time}} & \multicolumn{1}{l|}{\textbf{Received Time}} & \multicolumn{1}{l|}{\textbf{Motor Time}} & \multicolumn{1}{c|}{\textbf{\begin{tabular}[c]{@{}c@{}}Response \\ Time\end{tabular}}} \\ \hline
      Stop & 0.065822 & 02:19:51.590 & 02:19:51.640 & 02:19:52.007 & 0.417 \\ \hline
      Stop & 0.058279 & 02:19:53.439 & 02:19:53.488 & 02:19:53.840 & 0.401 \\ \hline
      Stop & 0.062168 & 02:19:56.840 & 02:19:56.870 & 02:19:57.281 & 0.441 \\ \hline
      Stop & 0.058816 & 02:19:59.188 & 02:19:59.220 & 02:19:59.587 & 0.399 \\ \hline
      Stop & 0.073755 & 02:20:01.641 & 02:20:01.672 & 02:20:02.038 & 0.397 \\ \hline
      Stop & 0.073311 & 02:20:04.506 & 02:20:04.557 & 02:20:04.920 & 0.414 \\ \hline
      Stop & 0.057240 & 02:20:06.989 & 02:20:07.040  & 02:20:07.407 & 0.418 \\ \hline
      Stop & 0.062809 & 02:20:09.857 & 02:20:09.882 & 02:20:10.238 & 0.381 \\ \hline
      Stop & 0.075541 & 02:34:15.546 & 02:34:15.572 & 02:34:15.943 & 0.397 \\ \hline
      Stop & 0.073136 & 02:34:18.263 & 02:34:18.290 & 02:34:18.658 & 0.395 \\ \hline
      Stop & 0.062294 & 02:34:19.139 & 02:34:19.339 & 02:34:19.674  & 0.535 \\ \hline
      Stop & 0.081555 & 02:34:20.662 & 02:34:20.705 & 02:34:21.058 & 0.396 \\ \hline
      Stop & 0.060251 & 02:34:25.540 & 02:34:25.571 & 02:34:25.956 & 0.416 \\ \hline
      Stop & 0.062418 & 02:34:28.455 & 02:34:28.488  & 02:34:28.856 & 0.401 \\ \hline
      Stop & 0.073785 & 02:34:31.504 & 02:34:31.589 & 02:34:31.971 & 0.467 \\ \hline
      Stop & 0.061575 & 02:34:34.440 & 02:34:34.487 & 02:34:34.839 & 0.399 \\ \hline
      Stop & 0.075160 & 02:34:37.389 & 02:34:37.422 & 02:34:37.807 & 0.418 \\ \hline
      Stop & 0.061302 & 02:34:40.406 & 02:34:40.439 & 02:34:40.822 & 0.416 \\ \hline
      Stop & 0.060768 & 02:34:43.656 & 02:34:43.708 & 02:34:44.057 & 0.401 \\ \hline
      Stop & 0.062792 & 02:34:46.491 & 02:34:46.524 & 02:34:46.906 & 0.415 \\ \hline
      Stop & 0.071784 & 02:34:49.839 & 02:34:49.873 & 02:34:50.255 & 0.416 \\ \hline
      Stop & 0.063085 & 03:01:41.475 & 03:01:41.527 & 03:01:41.910 & 0.435 \\ \hline
      Stop & 0.061328 & 03:01:43.691 & 03:01:43.810 & 03:01:44.172 & 0.481 \\ \hline
      Stop & 0.066716 & 03:01:46.175 & 03:01:46.206 & 03:01:46.571 & 0.396 \\ \hline
      Stop & 0.066698 & 03:01:46.841 & 03:01:47.260 & 03:01:47.626 & 0.785 \\ \hline
      Stop & 0.071177 & 03:01:49.675 & 03:01:49.711 & 03:01:50.074 & 0.399 \\ \hline
      Stop & 0.072667 & 03:01:51.373 & 03:01:51.395 & 03:01:51.788 & 0.415 \\ \hline
      Stop & 0.057458 & 03:01:52.692 & 03:01:52.760 & 03:01:53.127 & 0.435 \\ \hline
      Stop & 0.055361 & 03:01:54.144 & 03:01:54.192 & 03:01:54.561 & 0.417 \\ \hline
      Stop & 0.074920 & 03:01:56.425 & 03:01:56.505 & 03:01:56.875 & 0.450 \\ \hline
\end{longtable}

Dari semua hasil pengujian pada masing-masing kelas seperti yang tertera pada tabel-tabel di atas, dapat dilihat bahwa perintah "Kiri" memiliki waktu respons terendah dengan rata-rata 0,0933 detik, menunjukkan bahwa sistem dapat merespons perintah "Kiri" dengan cepat dan akurat. Sementara itu, perintah "Maju" dan "Stop" memiliki waktu respons yang lebih tinggi, masing-masing 0,4337 detik dan 0,4318 detik, menunjukkan bahwa sistem memerlukan waktu lebih lama untuk merespons perintah tersebut. Hal tersebut dikarenakan perintah "Maju" memiliki nilai \emph{Pulse Width Modulation} (PWM) yang lebih besar dibandingkan perintah lainnya.

\emph{Inference Time} untuk semua perintah berada pada kisaran yang sama, antara 0,0626 hingga 0,0666 detik, menunjukkan bahwa model dapat mendeteksi dan mengklasifikasikan gerakan mata dengan cepat dan konsisten di semua kelas. Secara keseluruhan, hasil ini menunjukkan bahwa sistem kontrol kursi roda berbasis gerakan mata memiliki kinerja yang andal dan responsif, terutama pada perintah "Kanan," "Kiri," dan "Mundur," yang memiliki waktu respons rata-rata di bawah 0,25 detik.

\section{Pengujian Kestabilan pada Motor Kursi Roda}

Pada skenario pengujian yang terakhir adalah pengujian kestabilan pada motor kursi roda. Pengujian kestabilan pada motor kursi roda bertujuan untuk memastikan bahwa waktu output yang dijalankan oleh motor tetap stabil untuk setiap input yang diberikan oleh pengguna melalui sistem kontrol gerakan mata. Stabilitas waktu output ini sangat penting untuk memberikan pengalaman penggunaan yang aman, nyaman, dan konsisten bagi pengguna kursi roda.

Pengujian ini melibatkan pengukuran waktu yang dibutuhkan oleh motor untuk merespons setiap perintah gerakan mata dan memastikan bahwa waktu respons tersebut tetap konsisten di berbagai kondisi input. Untuk memastikan hasil yang representatif, pengujian dilakukan per kelas perintah (seperti "Kanan," "Kiri," "Maju," "Mundur," dan "Stop") dengan masing-masing kelas diuji sebanyak 30 kali. Hasil pengujian akan dianalisis untuk menentukan tingkat kestabilan waktu respons motor dalam merespons perintah dari model kontrol gerakan mata. 

\subsection{Kestabilan Motor terhadap Kelas Kanan}

Variasi pengujian pertama yang dilakukan adalah pengujian kestabilan motor terhadap perintah "Kanan." Pengujian ini bertujuan untuk mengevaluasi apakah waktu output yang dijalankan oleh motor tetap stabil setiap kali perintah "Kanan" diberikan oleh pengguna melalui sistem kontrol gerakan mata. Hasil pengujian dapat dilihat pada Tabel \ref{tb:motorkanan} di bawah, dimana waktu output motor (\emph{Motor Time}) diukur untuk setiap input yang diberikan oleh pengguna melalui sistem kontrol gerakan mata.

Waktu output rata-rata motor untuk perintah "Kanan" adalah 6,013 detik, dengan variabilitas waktu berkisar antara 5,439 hingga 7,163 detik. Rentang waktu yang relatif sempit menunjukkan bahwa motor kursi roda dapat merespons perintah "Kanan" dengan cukup stabil dan konsisten. Secara keseluruhan, pengujian kestabilan motor kursi roda menunjukkan bahwa sistem kontrol mampu memberikan waktu respons yang stabil dan konsisten untuk perintah "Kanan." Meskipun ada sedikit variabilitas dalam waktu output motor, hasil ini menunjukkan bahwa sistem kontrol kursi roda berbasis gerakan mata dapat merespons perintah "Kanan" dengan cepat dan akurat.

%Tabel 4.23
\begin{longtable}{|c|c|c|}
  \caption{Hasil Pengujian Kestabilan Motor pada Kelas Kanan} 
  \label{tb:motorkanan} \\
  \hline
  \rowcolor[HTML]{C0C0C0} 
  \textbf{Kelas} & \textbf{Motor Time} & \textbf{Detection Time} \\ \hline
  Kanan          & 19:45:49.067        & \multirow{2}{*}{6.158}  \\ \cline{1-2}
  Stop           & 19:45:55.225        &                         \\ \hline
  Kanan          & 19:45:58.310        & \multirow{2}{*}{6.800}  \\ \cline{1-2}
  Stop           & 19:46:05.110        &                         \\ \hline
  Kanan          & 19:46:07.497        & \multirow{2}{*}{5.609}  \\ \cline{1-2}
  Stop           & 19:46:13.106        &                         \\ \hline
  Kanan          & 19:46:15.813        & \multirow{2}{*}{6.113}  \\ \cline{1-2}
  Stop           & 19:46:21.926        &                         \\ \hline
  Kanan          & 19:46:24.677        & \multirow{2}{*}{6.624}  \\ \cline{1-2}
  Stop           & 19:46:31.301        &                         \\ \hline
  Kanan          & 19:46:34.200        & \multirow{2}{*}{6.454}  \\ \cline{1-2}
  Stop           & 19:46:40.654        &                         \\ \hline
  Kanan          & 19:46:43.208        & \multirow{2}{*}{6.344}  \\ \cline{1-2}
  Stop           & 19:46:49.552        &                         \\ \hline
  Kanan          & 19:46:52.829        & \multirow{2}{*}{7.163}  \\ \cline{1-2}
  Stop           & 19:46:59.992        &                         \\ \hline
  Kanan          & 19:47:03.195        & \multirow{2}{*}{6.317}  \\ \cline{1-2}
  Stop           & 19:47:09.512        &                         \\ \hline
  Kanan          & 19:47:12.372        & \multirow{2}{*}{5.892}  \\ \cline{1-2}
  Stop           & 19:47:18.264        &                         \\ \hline
  Kanan          & 19:47:20.982        & \multirow{2}{*}{6.001}  \\ \cline{1-2}
  Stop           & 19:47:26.983        &                         \\ \hline
  Kanan          & 19:47:29.702        & \multirow{2}{*}{5.531}  \\ \cline{1-2}
  Stop           & 19:47:35.233        &                         \\ \hline
  Kanan          & 19:47:37.858        & \multirow{2}{*}{6.142}  \\ \cline{1-2}
  Stop           & 19:47:44.000        &                         \\ \hline
  Kanan          & 19:47:46.495        & \multirow{2}{*}{6.045}  \\ \cline{1-2}
  Stop           & 19:47:52.540        &                         \\ \hline
  Kanan          & 19:47:55.400        & \multirow{2}{*}{6.000}  \\ \cline{1-2}
  Stop           & 19:48:01.400        &                         \\ \hline
  Kanan          & 19:48:03.979        & \multirow{2}{*}{5.907}  \\ \cline{1-2}
  Stop           & 19:48:09.886        &                         \\ \hline
  Kanan          & 19:48:12.839        & \multirow{2}{*}{5.566}  \\ \cline{1-2}
  Stop           & 19:48:18.405        &                         \\ \hline
  Kanan          & 19:48:21.171        & \multirow{2}{*}{5.532}  \\ \cline{1-2}
  Stop           & 19:48:26.703        &                         \\ \hline
  Kanan          & 19:48:29.610        & \multirow{2}{*}{5.672}  \\ \cline{1-2}
  Stop           & 19:48:35.282        &                         \\ \hline
  Kanan          & 19:48:37.908        & \multirow{2}{*}{6.000}  \\ \cline{1-2}
  Stop           & 19:48:43.908        &                         \\ \hline
  Kanan          & 19:48:46.580        & \multirow{2}{*}{6.047}  \\ \cline{1-2}
  Stop           & 19:48:52.627        &                         \\ \hline
  Kanan          & 19:48:55.158        & \multirow{2}{*}{5.813}  \\ \cline{1-2}
  Stop           & 19:49:00.971        &                         \\ \hline
  Kanan          & 19:49:03.503        & \multirow{2}{*}{5.578}  \\ \cline{1-2}
  Stop           & 19:49:09.081        &                         \\ \hline
  Kanan          & 19:49:11.799        & \multirow{2}{*}{5.980}  \\ \cline{1-2}
  Stop           & 19:49:17.779        &                         \\ \hline
  Kanan          & 19:49:20.358        & \multirow{2}{*}{5.871}  \\ \cline{1-2}
  Stop           & 19:49:26.229        &                         \\ \hline
  Kanan          & 19:49:28.948        & \multirow{2}{*}{5.860}  \\ \cline{1-2}
  Stop           & 19:49:34.808        &                         \\ \hline
  Kanan          & 19:49:37.435        & \multirow{2}{*}{5.625}  \\ \cline{1-2}
  Stop           & 19:49:43.060        &                         \\ \hline
  Kanan          & 19:49:45.732        & \multirow{2}{*}{5.768}  \\ \cline{1-2}
  Stop           & 19:49:51.500        &                         \\ \hline
  Kanan          & 19:49:54.034        & \multirow{2}{*}{5.439}  \\ \cline{1-2}
  Stop           & 19:49:59.473        &                         \\ \hline
  Kanan          & 19:50:02.098        & \multirow{2}{*}{6.546}  \\ \cline{1-2}
  Stop           & 19:50:08.644        &                         \\ \hline
\end{longtable}

\subsection{Kestabilan Motor terhadap Perintah Kiri}

Selanjutnya dilakukan pengujian kestabilan motor terhadap perintah "Kiri." Tujuan dari pengujian ini adalah untuk mengevaluasi apakah waktu output yang dijalankan oleh motor tetap stabil dan konsisten setiap kali perintah "Kiri" diberikan oleh pengguna melalui sistem kontrol gerakan mata. Hasil pengujian dapat dilihat pada Tabel \ref{tb:motorkiri} di bawah. Dapat diamati bahwa waktu output rata-rata motor untuk perintah "Kiri" adalah 6,469 detik, dengan rentang waktu mulai dari 5,931 hingga 8,032 detik. 

Sebagian besar waktu output motor berada dalam kisaran 6 detik, menunjukkan bahwa motor kursi roda dapat merespons perintah "Kiri" dengan tingkat stabilitas dan konsistensi yang baik.  Namun, terdapat beberapa variabilitas yang terlihat, seperti waktu output motor mencapai 8,032 detik. Secara keseluruhan, pengujian ini menunjukkan bahwa sistem kontrol kursi roda mampu memberikan waktu respons yang cukup stabil untuk perintah "Kiri".

%Tabel 4.24
\begin{longtable}{|c|c|c|}
  \caption{Hasil Pengujian Kestabilan Motor pada Kelas Kiri} 
  \label{tb:motorkiri} \\
  \hline
  \rowcolor[HTML]{C0C0C0} 
  \textbf{Kelas} & \textbf{Motor Time} & \textbf{Detection Time} \\ \hline
  Kiri           & 20:16:00.873        & \multirow{2}{*}{6.546}  \\ \cline{1-2}
  Stop           & 20:16:07.419        &                         \\ \hline
  Kiri           & 20:16:09.819        & \multirow{2}{*}{6.751}  \\ \cline{1-2}
  Stop           & 20:16:16.570        &                         \\ \hline
  Kiri           & 20:16:18.719        & \multirow{2}{*}{6.133}  \\ \cline{1-2}
  Stop           & 20:16:24.852        &                         \\ \hline
  Kiri           & 20:16:27.635        & \multirow{2}{*}{6.435}  \\ \cline{1-2}
  Stop           & 20:16:34.070        &                         \\ \hline
  Kiri           & 20:16:36.936        & \multirow{2}{*}{6.350}  \\ \cline{1-2}
  Stop           & 20:16:43.286        &                         \\ \hline
  Kiri           & 20:16:45.419        & \multirow{2}{*}{6.516}  \\ \cline{1-2}
  Stop           & 20:16:51.935        &                         \\ \hline
  Kiri           & 20:16:53.936        & \multirow{2}{*}{6.800}  \\ \cline{1-2}
  Stop           & 20:17:00.736        &                         \\ \hline
  Kiri           & 20:17:03.252        & \multirow{2}{*}{6.467}  \\ \cline{1-2}
  Stop           & 20:17:09.719        &                         \\ \hline
  Kiri           & 20:17:11.620        & \multirow{2}{*}{6.217}  \\ \cline{1-2}
  Stop           & 20:17:17.837        &                         \\ \hline
  Kiri           & 20:17:20.304        & \multirow{2}{*}{6.867}  \\ \cline{1-2}
  Stop           & 20:17:27.171        &                         \\ \hline
  Kiri           & 20:17:29.369        & \multirow{2}{*}{6.318}  \\ \cline{1-2}
  Stop           & 20:17:35.687        &                         \\ \hline
  Kiri           & 20:17:37.604        & \multirow{2}{*}{6.468}  \\ \cline{1-2}
  Stop           & 20:17:44.072        &                         \\ \hline
  Kiri           & 20:17:45.988        & \multirow{2}{*}{6.398}  \\ \cline{1-2}
  Stop           & 20:17:52.386        &                         \\ \hline
  Kiri           & 20:17:54.387        & \multirow{2}{*}{6.433}  \\ \cline{1-2}
  Stop           & 20:18:00.820        &                         \\ \hline
  Kiri           & 20:18:03.504        & \multirow{2}{*}{8.032}  \\ \cline{1-2}
  Stop           & 20:18:11.536        &                         \\ \hline
  Kiri           & 20:18:13.904        & \multirow{2}{*}{6.684}  \\ \cline{1-2}
  Stop           & 20:18:20.588        &                         \\ \hline
  Kiri           & 20:18:22.387        & \multirow{2}{*}{6.751}  \\ \cline{1-2}
  Stop           & 20:18:29.138        &                         \\ \hline
  Kiri           & 20:18:31.204        & \multirow{2}{*}{6.132}  \\ \cline{1-2}
  Stop           & 20:18:37.336        &                         \\ \hline
  Kiri           & 20:18:39.486        & \multirow{2}{*}{6.517}  \\ \cline{1-2}
  Stop           & 20:18:46.003        &                         \\ \hline
  Kiri           & 20:18:48.203        & \multirow{2}{*}{6.450}  \\ \cline{1-2}
  Stop           & 20:18:54.653        &                         \\ \hline
  Kiri           & 20:18:56.920        & \multirow{2}{*}{6.350}  \\ \cline{1-2}
  Stop           & 20:19:03.270        &                         \\ \hline
  Kiri           & 20:19:05.337        & \multirow{2}{*}{6.516}  \\ \cline{1-2}
  Stop           & 20:19:11.853        &                         \\ \hline
  Kiri           & 20:19:13.836        & \multirow{2}{*}{6.251}  \\ \cline{1-2}
  Stop           & 20:19:20.087        &                         \\ \hline
  Kiri           & 20:19:22.204        & \multirow{2}{*}{6.099}  \\ \cline{1-2}
  Stop           & 20:19:28.303        &                         \\ \hline
  Kiri           & 20:19:30.770        & \multirow{2}{*}{6.501}  \\ \cline{1-2}
  Stop           & 20:19:37.271        &                         \\ \hline
  Kiri           & 20:19:39.473        & \multirow{2}{*}{5.931}  \\ \cline{1-2}
  Stop           & 20:19:45.404        &                         \\ \hline
  Kiri           & 20:19:47.272        & \multirow{2}{*}{6.166}  \\ \cline{1-2}
  Stop           & 20:19:53.438        &                         \\ \hline
  Kiri           & 20:19:55.302        & \multirow{2}{*}{6.185}  \\ \cline{1-2}
  Stop           & 20:20:01.487        &                         \\ \hline
  Kiri           & 20:20:03.506        & \multirow{2}{*}{6.597}  \\ \cline{1-2}
  Stop           & 20:20:10.103        &                         \\ \hline
  Kiri           & 20:20:11.703        & \multirow{2}{*}{6.217}  \\ \cline{1-2}
  Stop           & 20:20:17.920        &                         \\ \hline
\end{longtable}

\subsection{Kestabilan Motor terhadap Perintah Maju}

Pengujian berikutnya adalah pengujian kestabilan motor terhadap perintah "Maju". Pengujian ini bertujuan untuk mengevaluasi apakah waktu output motor tetap konsisten setiap kali pengguna memberikan perintah "Maju" melalui sistem kontrol gerakan mata. 

Tabel \ref{tb:motormaju} di bawah menunjukkan hasil pengujian kestabilan motor kursi roda untuk perintah "Maju." Waktu rata-rata motor untuk perintah "Maju" adalah 6,532 detik, dengan rentang antara 6,285 hingga 6,796 detik. Sebagian besar waktu output motor berada di kisaran 6,4 hingga 6,6 detik, menunjukkan bahwa motor kursi roda dapat merespons perintah "Maju" dengan cukup stabil dan konsisten.

Secara keseluruhan, sistem kontrol kursi roda mampu memberikan waktu respons yang stabil untuk perintah "Maju." Meskipun ada beberapa variasi dalam waktu output, hasil ini menunjukkan bahwa sistem dapat merespons perintah "Maju" dengan cukup baik.

%Tabel 4.25
\begin{longtable}{|c|c|c|}
  \caption{Hasil Pengujian Kestabilan Motor pada Kelas Maju}  
  \label{tb:motormaju} \\
  \hline
  \rowcolor[HTML]{C0C0C0} 
  \textbf{Kelas} & \textbf{Motor Time} & \textbf{Detection Time} \\ \hline
  Maju           & 18:01:35.901        & \multirow{2}{*}{6.500}  \\ \cline{1-2}
  Stop           & 18:01:42.401        &                         \\ \hline
  Maju           & 18:01:44.533        & \multirow{2}{*}{6.285}  \\ \cline{1-2}
  Stop           & 18:01:50.818        &                         \\ \hline
  Maju           & 18:01:52.803        & \multirow{2}{*}{6.345}  \\ \cline{1-2}
  Stop           & 18:01:59.148        &                         \\ \hline
  Maju           & 18:02:01.248        & \multirow{2}{*}{6.565}  \\ \cline{1-2}
  Stop           & 18:02:07.813        &                         \\ \hline
  Maju           & 18:02:09.566        & \multirow{2}{*}{6.489}  \\ \cline{1-2}
  Stop           & 18:02:16.055        &                         \\ \hline
  Maju           & 18:02:17.917        & \multirow{2}{*}{6.570}  \\ \cline{1-2}
  Stop           & 18:02:24.487        &                         \\ \hline
  Maju           & 18:02:26.317        & \multirow{2}{*}{6.784}  \\ \cline{1-2}
  Stop           & 18:02:33.101        &                         \\ \hline
  Maju           & 18:02:35.287        & \multirow{2}{*}{6.581}  \\ \cline{1-2}
  Stop           & 18:02:41.868        &                         \\ \hline
  Maju           & 18:02:43.683        & \multirow{2}{*}{6.717}  \\ \cline{1-2}
  Stop           & 18:02:50.400        &                         \\ \hline
  Maju           & 18:02:52.256        & \multirow{2}{*}{6.576}  \\ \cline{1-2}
  Stop           & 18:02:58.832        &                         \\ \hline
  Maju           & 18:03:00.568        & \multirow{2}{*}{6.480}  \\ \cline{1-2}
  Stop           & 18:03:07.048        &                         \\ \hline
  Maju           & 18:03:08.981        & \multirow{2}{*}{6.457}  \\ \cline{1-2}
  Stop           & 18:03:15.438        &                         \\ \hline
  Maju           & 18:03:17.232        & \multirow{2}{*}{6.618}  \\ \cline{1-2}
  Stop           & 18:03:23.850        &                         \\ \hline
  Maju           & 18:03:25.714        & \multirow{2}{*}{6.436}  \\ \cline{1-2}
  Stop           & 18:03:32.150        &                         \\ \hline
  Maju           & 18:03:33.968        & \multirow{2}{*}{6.417}  \\ \cline{1-2}
  Stop           & 18:03:40.385        &                         \\ \hline
  Maju           & 18:03:42.173        & \multirow{2}{*}{6.595}  \\ \cline{1-2}
  Stop           & 18:03:48.768        &                         \\ \hline
  Maju           & 18:03:50.568        & \multirow{2}{*}{6.515}  \\ \cline{1-2}
  Stop           & 18:03:57.083        &                         \\ \hline
  Maju           & 18:03:58.934        & \multirow{2}{*}{6.450}  \\ \cline{1-2}
  Stop           & 18:04:05.384        &                         \\ \hline
  Maju           & 18:04:07.235        & \multirow{2}{*}{6.599}  \\ \cline{1-2}
  Stop           & 18:04:13.834        &                         \\ \hline
  Maju           & 18:04:15.620        & \multirow{2}{*}{6.530}  \\ \cline{1-2}
  Stop           & 18:04:22.150        &                         \\ \hline
  Maju           & 18:04:24.120        & \multirow{2}{*}{6.619}  \\ \cline{1-2}
  Stop           & 18:04:30.739        &                         \\ \hline
  Maju           & 18:04:32.899        & \multirow{2}{*}{6.669}  \\ \cline{1-2}
  Stop           & 18:04:39.568        &                         \\ \hline
  Maju           & 18:04:41.433        & \multirow{2}{*}{6.582}  \\ \cline{1-2}
  Stop           & 18:04:48.015        &                         \\ \hline
  Maju           & 18:04:49.819        & \multirow{2}{*}{6.498}  \\ \cline{1-2}
  Stop           & 18:04:56.317        &                         \\ \hline
  Maju           & 18:04:58.207        & \multirow{2}{*}{6.796}  \\ \cline{1-2}
  Stop           & 18:05:05.003        &                         \\ \hline
  Maju           & 18:05:06.768        & \multirow{2}{*}{6.540}  \\ \cline{1-2}
  Stop           & 18:05:13.308        &                         \\ \hline
  Maju           & 18:05:15.418        & \multirow{2}{*}{6.500}  \\ \cline{1-2}
  Stop           & 18:05:21.918        &                         \\ \hline
  Maju           & 18:05:23.734        & \multirow{2}{*}{6.386}  \\ \cline{1-2}
  Stop           & 18:05:30.120        &                         \\ \hline
  Maju           & 18:05:31.967        & \multirow{2}{*}{6.421}  \\ \cline{1-2}
  Stop           & 18:05:38.388        &                         \\ \hline
  Maju           & 18:05:40.171        & \multirow{2}{*}{6.431}  \\ \cline{1-2}
  Stop           & 18:05:46.602        &                         \\ \hline
\end{longtable}

\subsection{Kestabilan Motor terhadap Perintah Mundur}

Selanjutnya, dilakukan pengujian untuk mengukur seberapa stabil motor kursi roda merespons perintah "Mundur." Tujuan pengujian ini adalah untuk memastikan bahwa waktu output motor tetap konsisten setiap kali pengguna memberikan perintah "Mundur" melalui sistem kontrol gerakan mata. 

Dari hasil pengujian yang tertera pada Tabel \ref{tb:motormundur} di bawah, waktu output rata-rata motor untuk perintah "Mundur" adalah 6,863 detik, dengan kisaran waktu antara 6,470 hingga 7,235 detik. Sebagian besar waktu output motor berada di kisaran 6,6 hingga 7 detik, menunjukkan bahwa motor kursi roda dapat merespons perintah "Mundur" dengan stabil dan konsisten.

%Tabel 4.26
\begin{longtable}{|c|c|c|}
  \caption{Hasil Pengujian Kestabilan Motor pada Kelas Mundur}  
  \label{tb:motormundur} \\
  \hline
  \rowcolor[HTML]{C0C0C0} 
  \textbf{Kelas} & \textbf{Motor Time} & \textbf{Detection Time} \\ \hline
  Mundur & 18:55:51.283 & \multirow{2}{*}{7.149} \\ \cline{1-2}
  Stop   & 18:55:58.432 &                        \\ \hline
  Mundur & 18:56:00.166 & \multirow{2}{*}{6.682} \\ \cline{1-2}
  Stop   & 18:56:06.848 &                        \\ \hline
  Mundur & 18:56:08.481 & \multirow{2}{*}{6.601} \\ \cline{1-2}
  Stop   & 18:56:15.082 &                        \\ \hline
  Mundur & 18:56:16.666 & \multirow{2}{*}{6.781} \\ \cline{1-2}
  Stop   & 18:56:23.447 &                        \\ \hline
  Mundur & 18:56:25.066 & \multirow{2}{*}{6.798} \\ \cline{1-2}
  Stop   & 18:56:31.864 &                        \\ \hline
  Mundur & 18:56:33.465 & \multirow{2}{*}{6.868} \\ \cline{1-2}
  Stop   & 18:56:40.333 &                        \\ \hline
  Mundur & 18:56:42.269 & \multirow{2}{*}{7.028} \\ \cline{1-2}
  Stop   & 18:56:49.297 &                        \\ \hline
  Mundur & 18:56:51.030 & \multirow{2}{*}{6.654} \\ \cline{1-2}
  Stop   & 18:56:57.684 &                        \\ \hline
  Mundur & 18:56:59.267 & \multirow{2}{*}{6.932} \\ \cline{1-2}
  Stop   & 18:57:06.199 &                        \\ \hline
  Mundur & 18:57:07.882 & \multirow{2}{*}{6.783} \\ \cline{1-2}
  Stop   & 18:57:14.665 &                        \\ \hline
  Mundur & 18:57:16.382 & \multirow{2}{*}{6.983} \\ \cline{1-2}
  Stop   & 18:57:23.365 &                        \\ \hline
  Mundur & 18:57:25.031 & \multirow{2}{*}{7.002} \\ \cline{1-2}
  Stop   & 18:57:32.033 &                        \\ \hline
  Mundur & 18:57:33.597 & \multirow{2}{*}{6.817} \\ \cline{1-2}
  Stop   & 18:57:40.414 &                        \\ \hline
  Mundur & 18:57:42.034 & \multirow{2}{*}{6.766} \\ \cline{1-2}
  Stop   & 18:57:48.800 &                        \\ \hline
  Mundur & 18:57:50.430 & \multirow{2}{*}{6.801} \\ \cline{1-2}
  Stop   & 18:57:57.231 &                        \\ \hline
  Mundur & 18:57:58.797 & \multirow{2}{*}{7.018} \\ \cline{1-2}
  Stop   & 18:58:05.815 &                        \\ \hline
  Mundur & 18:58:07.447 & \multirow{2}{*}{6.786} \\ \cline{1-2}
  Stop   & 18:58:14.233 &                        \\ \hline
  Mundur & 18:58:16.066 & \multirow{2}{*}{6.964} \\ \cline{1-2}
  Stop   & 18:58:23.030 &                        \\ \hline
  Mundur & 18:58:24.598 & \multirow{2}{*}{7.235} \\ \cline{1-2}
  Stop   & 18:58:31.833 &                        \\ \hline
  Mundur & 18:58:33.498 & \multirow{2}{*}{6.851} \\ \cline{1-2}
  Stop   & 18:58:40.349 &                        \\ \hline
  Mundur & 18:58:42.064 & \multirow{2}{*}{6.948} \\ \cline{1-2}
  Stop   & 18:58:49.012 &                        \\ \hline
  Mundur & 18:58:50.901 & \multirow{2}{*}{6.780} \\ \cline{1-2}
  Stop   & 18:58:57.681 &                        \\ \hline
  Mundur & 18:58:59.331 & \multirow{2}{*}{6.819} \\ \cline{1-2}
  Stop   & 18:59:06.150 &                        \\ \hline
  Mundur & 18:59:07.931 & \multirow{2}{*}{6.950} \\ \cline{1-2}
  Stop   & 18:59:14.881 &                        \\ \hline
  Mundur & 18:59:16.615 & \multirow{2}{*}{6.767} \\ \cline{1-2}
  Stop   & 18:59:23.382 &                        \\ \hline
  Mundur & 18:59:24.828 & \multirow{2}{*}{6.972} \\ \cline{1-2}
  Stop   & 18:59:31.800 &                        \\ \hline
  Mundur & 18:59:33.348 & \multirow{2}{*}{6.852} \\ \cline{1-2}
  Stop   & 18:59:40.200 &                        \\ \hline
  Mundur & 18:59:41.780 & \multirow{2}{*}{6.470} \\ \cline{1-2}
  Stop   & 18:59:48.250 &                        \\ \hline
  Mundur & 18:59:49.915 & \multirow{2}{*}{6.868} \\ \cline{1-2}
  Stop   & 18:59:56.783 &                        \\ \hline
  Mundur & 18:59:58.397 & \multirow{2}{*}{6.969} \\ \cline{1-2}
  Stop   & 19:00:05.366 &                        \\ \hline
\end{longtable}

\subsection{Kestabilan Motor terhadap Perintah Stop}

Pengujian yang terakhir adalah pengujian kestabilan motor terhadap perintah "Stop". Perintah "Stop" memiliki peran krusial dalam menghentikan pergerakan kursi roda secara tepat waktu dan akurat. Tujuan pengujian ini adalah untuk mengukur waktu respons motor saat diberikan perintah "Stop," serta memastikan bahwa waktu output motor tetap stabil dan konsisten.

Berdasarkan pengamatan dari Tabel \ref{tb:motorstop} di bawah, waktu output rata-rata motor untuk perintah "Stop" adalah sekitar 4,933 detik, dengan kisaran antara 4,294 hingga 5,785 detik. Sebagian besar waktu output berada di kisaran 4,4 hingga 5,3 detik, menunjukkan bahwa motor merespons perintah "Stop" dengan cukup stabil dan konsisten.

%Tabel 4.27
\begin{longtable}{|c|c|c|}
  \caption{Hasil Pengujian Kestabilan Motor pada Kelas Stop}  
  \label{tb:motorstop} \\
  \hline
  \rowcolor[HTML]{C0C0C0} 
  \textbf{Kelas} & \textbf{Motor Time} & \textbf{Detection Time} \\ \hline
  Stop           & 16:51:16.866        & \multirow{2}{*}{4,952}  \\ \cline{1-2}
  Maju           & 16:51:21.818        &                         \\ \hline
  Stop           & 16:51:26.379        & \multirow{2}{*}{5,039}  \\ \cline{1-2}
  Maju           & 16:51:31.418        &                         \\ \hline
  Stop           & 16:51:35.931        & \multirow{2}{*}{4.74}   \\ \cline{1-2}
  Maju           & 16:51:40.671        &                         \\ \hline
  Stop           & 16:51:45.249        & \multirow{2}{*}{4,533}  \\ \cline{1-2}
  Maju           & 16:51:49.782        &                         \\ \hline
  Stop           & 16:51:54.480        & \multirow{2}{*}{5,299}  \\ \cline{1-2}
  Maju           & 16:51:59.779        &                         \\ \hline
  Stop           & 16:52:04.266        & \multirow{2}{*}{4,932}  \\ \cline{1-2}
  Maju           & 16:52:09.198        &                         \\ \hline
  Stop           & 16:52:13.683        & \multirow{2}{*}{5,296}  \\ \cline{1-2}
  Maju           & 16:52:18.979        &                         \\ \hline
  Stop           & 16:52:23.832        & \multirow{2}{*}{4,963}  \\ \cline{1-2}
  Maju           & 16:52:28.795        &                         \\ \hline
  Stop           & 16:52:33.913        & \multirow{2}{*}{5.02}   \\ \cline{1-2}
  Maju           & 16:52:38.933        &                         \\ \hline
  Stop           & 16:52:43.883        & \multirow{2}{*}{4,981}  \\ \cline{1-2}
  Maju           & 16:52:48.864        &                         \\ \hline
  Stop           & 16:52:53.335        & \multirow{2}{*}{5,295}  \\ \cline{1-2}
  Maju           & 16:52:58.630        &                         \\ \hline
  Stop           & 16:53:03.496        & \multirow{2}{*}{5,067}  \\ \cline{1-2}
  Maju           & 16:53:08.563        &                         \\ \hline
  Stop           & 16:53:13.379        & \multirow{2}{*}{4,906}  \\ \cline{1-2}
  Maju           & 16:53:18.285        &                         \\ \hline
  Stop           & 16:53:23.319        & \multirow{2}{*}{4,781}  \\ \cline{1-2}
  Maju           & 16:53:28.100        &                         \\ \hline
  Stop           & 16:53:32.816        & \multirow{2}{*}{5,283}  \\ \cline{1-2}
  Maju           & 16:53:38.099        &                         \\ \hline
  Stop           & 16:53:42.950        & \multirow{2}{*}{4,883}  \\ \cline{1-2}
  Maju           & 16:53:47.833        &                         \\ \hline
  Stop           & 16:53:52.261        & \multirow{2}{*}{5,041}  \\ \cline{1-2}
  Maju           & 16:53:57.302        &                         \\ \hline
  Stop           & 16:54:02.598        & \multirow{2}{*}{5,551}  \\ \cline{1-2}
  Maju           & 16:54:08.149        &                         \\ \hline
  Stop           & 17:31:56.140        & \multirow{2}{*}{4.85}   \\ \cline{1-2}
  Maju           & 17:32:00.990        &                         \\ \hline
  Stop           & 17:32:05.123        & \multirow{2}{*}{4,294}  \\ \cline{1-2}
  Maju           & 17:32:09.417        &                         \\ \hline
  Stop           & 17:32:13.623        & \multirow{2}{*}{4,658}  \\ \cline{1-2}
  Maju           & 17:32:18.281        &                         \\ \hline
  Stop           & 17:32:22.684        & \multirow{2}{*}{4,638}  \\ \cline{1-2}
  Maju           & 17:32:27.322        &                         \\ \hline
  Stop           & 17:32:31.462        & \multirow{2}{*}{4,465}  \\ \cline{1-2}
  Maju           & 17:32:35.927        &                         \\ \hline
  Stop           & 17:32:40.077        & \multirow{2}{*}{4,396}  \\ \cline{1-2}
  Maju           & 17:32:44.473        &                         \\ \hline
  Stop           & 17:32:48.824        & \multirow{2}{*}{5.32}   \\ \cline{1-2}
  Maju           & 17:32:54.144        &                         \\ \hline
  Stop           & 17:32:59.260        & \multirow{2}{*}{5,118}  \\ \cline{1-2}
  Maju           & 17:33:04.378        &                         \\ \hline
  Stop           & 17:33:08.959        & \multirow{2}{*}{5,785}  \\ \cline{1-2}
  Maju           & 17:33:14.744        &                         \\ \hline
  Stop           & 17:33:18.943        & \multirow{2}{*}{5,265}  \\ \cline{1-2}
  Maju           & 17:33:24.208        &                         \\ \hline
  Stop           & 17:33:28.725        & \multirow{2}{*}{4,433}  \\ \cline{1-2}
  Maju           & 17:33:33.158        &                         \\ \hline
  Stop           & 17:33:37.263        & \multirow{2}{*}{4,414}  \\ \cline{1-2}
  Maju           & 17:33:41.677        &                         \\ \hline
\end{longtable}

Pengujian kestabilan motor kursi roda terhadap berbagai perintah ("Kanan," "Kiri," "Maju," "Mundur," dan "Stop") menunjukkan tingkat konsistensi yang bervariasi. Waktu output rata-rata untuk perintah "Kanan" adalah 6,013 detik, dengan rentang 5,439 hingga 7,163 detik, menandakan bahwa motor merespons perintah ini dengan cukup stabil meski terdapat beberapa variasi. Untuk perintah "Kiri," waktu output rata-rata adalah 6,469 detik, dengan kisaran 5,931 hingga 8,032 detik, menunjukkan adanya variasi yang lebih besar dalam respons motor.

Perintah "Maju" memiliki waktu output rata-rata 6,532 detik, dengan rentang antara 6,285 hingga 6,796 detik, menunjukkan konsistensi yang baik dalam kisaran 6 detik. Sementara itu, perintah "Mundur" menunjukkan waktu output rata-rata 6,863 detik, dengan variasi dari 6,470 hingga 7,235 detik, menandakan perlunya peningkatan stabilitas respons. Perintah "Stop" memiliki waktu output rata-rata tercepat, yaitu 4,933 detik, dengan kisaran antara 4,294 hingga 5,785 detik, menunjukkan respons motor yang lebih cepat dibandingkan perintah lainnya, meskipun terdapat beberapa variasi.

Secara keseluruhan, sistem kontrol kursi roda mampu merespons semua perintah dengan tingkat stabilitas yang cukup baik. Perintah "Maju" menunjukkan stabilitas respons terbaik dengan rentang waktu output yang sempit, sementara perintah "Kiri" memiliki variasi waktu output terbesar. Meskipun terdapat variasi kecil dalam waktu output pada masing-masing kelas, hasil pengujian ini menunjukkan bahwa motor kursi roda merespons perintah dengan cukup konsisten.