\begin{center}
  \Large
  \textbf{KATA PENGANTAR}
\end{center}

\addcontentsline{toc}{chapter}{KATA PENGANTAR}

\vspace{2ex}

% Ubah paragraf-paragraf berikut dengan isi dari kata pengantar

Puji dan syukur ke hadirat Tuhan Yang Maha Esa, atas segala rahmat dan karunia-Nya,
sehingga penulis dapat menyelesaikan penelitian ini yang berjudul
"\tatitle".

Penelitian ini disusun dalam rangka pemenuhan Tugas Akhir sebagai syarat kelulusan sehingga mendapatkan gelar Sarjana Teknik komputer ITS. Oleh karena itu, penulis mengucapkan banyak terima kasih kepada:

\begin{enumerate}[nolistsep]
  \item Tuhan Yesus Kristus atas segala Berkat dan Anugerah-Nya sehingga penulis dapat menyelesaikan masa kuliah dan mengerjakan tugas akhir ini dengan baik dan lancar,

  \item Kedua Orang Tua, Papi dan Mami, yang selalu memberikan doa, dukungan, motivasi, nasihat, dan uang saku, serta adik-adik yang telah memberi penulis semangat,
  
  \item Kakek, Nenek, Opung, beserta seluruh keluarga besar yang selalu memberikan dukungan, doa, nasihat, dan semangat kepada penulis,

  \item Bapak Dr. Supeno Mardi Susiko Nugroho, ST.,MT, selaku Kepala Departemen Teknik Komputer, Fakultas Teknologi Elektro dan Informatika Cerdas, Institut Teknologi Sepuluh Nopember,

  \item Bapak Dr. Eko Mulyanto Yuniarno, S.T., M.T. selaku Dosen Pembimbing I yang telah memberikan topik, bimbingan, wawasan, dan kesempatan untuk mengerjakan tugas akhir ini,

  \item Bapak Dion Hayu Fandiantoro, S.T., M.Eng. selaku Dosen Pembimbing II yang telah memberikan banyak masukan dan revisi seputar pengujian dan penulisan buku tugas akhir ini,

  \item Bapak-Ibu dosen pengajar Departemen Teknik Komputer atas ilmu dan pengajaran yang telah diberikan kepada penulis selama kuliah,
  
  \item Teman - teman sesama bimbingan tugas akhir terutama kursi roda, teman-teman dari lab MIOT, Telematika, dan Robotika serta teman - teman Departemen Teknik Komputer lainnya yang telah membantu dan menemani penulis dalam menjalani masa kuliah.

\end{enumerate}

Akhir kata, semoga penelitian ini bermanfaat bagi banyak pihak. Penulis menyadari bahwa tugas akhir ini masih jauh dari sempurna karena keterbatasan ilmu. Oleh karena itu, penulis menerima kritik dan saran yang membangun guna perbaikan di masa yang akan datang.

\begin{flushright}
  \begin{tabular}[b]{c}
    \place{}, \MONTH{} \the\year{} \\
    \\
    \\
    \\
    \\
    \name{}
  \end{tabular}
\end{flushright}